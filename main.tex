
\documentclass[master]{thesis-uestc}
\usepackage{listings}  % 引入 listings 宏包,用于显示代码
\usepackage{enumitem}
\usepackage{placeins}  % 导言区添加
\title{面向长尾分布数据的自监督学习故障诊断方法研究}{}

% \author{王本浩}{Wang Benhao}
% \advisor{*\chinesespace 教授}{Dr. Xue Zhou}
% \school{电子科技大学(深圳)高等研究院}{Shenzhen Institute for Advanced Study, UESTC}
% \major{电子信息}{Electronic Information}
% \studentnumber{202222280517}
\author{王本浩}{Wang Benhao}
\advisor{*\chinesespace *}{*}
\school{电子科技大学(深圳)高等研究院}{Shenzhen Institute for Advanced Study, UESTC}
\major{电子信息}{Electronic Information}
\studentnumber{202222280517}
% \author{*}{*}
% \advisor{*\chinesespace 教授}{*}
% \school{电子科技大学(深圳)高等研究院}{Shenzhen Institute for Advanced Study, UESTC}
% \major{电子信息}{Electronic Information}
% \studentnumber{*}
\begin{document}

\makecover

\begin{chineseabstract}
    随着工业系统的不断复杂化,系统中可能出现的故障类型也越来越多,导致故障诊断数据集常呈现“长尾分布”,即大多数样本集中在少数几类(头部类),而其他多数类别(尾部类)样本稀少。这种不平衡会使模型倾向于识别头部类,而忽视尾部类。然而,尾部类常包含危害更大的罕见故障,因此在故障诊断中更具价值。为应对这一问题,自监督学习成为主流方法,其训练分为两个阶段:首先通过无标签数据进行特征提取预训练,再在有标签数据上进行微调分类。

    本文旨在研究面向长尾分布数据的自监督学习故障诊断方法,主要内容包括:(1)传统的时域信号增强策略在训练简单暹罗孪生网络以进行故障诊断时,常难以有效提升其特征表征能力。本文从理论层面分析了该类网络中数据增强模块应具备的核心特性,进而提出了一种基于协方差矩阵适应进化策略的数据增强策略搜索算法。该算法通过在小规模数据上的迭代搜索得到一个最适用当前数据对象的增强策略参数,减轻了人工设计数据增强模块的时间成本。该算法训练的编码器层与SOTA算法SimCLR和BYOL相比有更加卓越的特征提取能力,并通过消融实验进一步验证了本算法在不同Batch Size设置下所表现出的稳定性与鲁棒性;(2)针对简单暹罗孪生网络在微调分类器层时更容易受到长尾效应的影响,从而分类准确率低于SimCLR的问题引入了MARC决策面校准算法。但MARC决策面校准算法并没有合理应用大规模无标签数据集中的信息,因此结合半监督学习的思想,将原模型预测的无监督数据的类别作为该数据的伪标签,并与原有标签数据集构成新的有标签数据集微调并校准网络。实验证明,本研究的方法在构建的长尾分布轴承故障诊断数据集上,成功超越了现有SOTA方法SimCLR和BYOL,取得了更优的性能。

\chinesekeyword{故障诊断,长尾学习,自监督学习,孪生网络}
\end{chineseabstract}

\begin{englishabstract}
    As modern industrial systems become increasingly complex, the number of fault types in a single system continues to grow, resulting in datasets that often exhibit a long-tailed distribution. Training models with long-tailed data introduces a significant class imbalance problem, and learning under long-tailed distributions is referred to as long-tail learning. Head classes typically refer to categories with a large number of samples, while tail classes refer to those with fewer samples. The goal of long-tail learning is to enable the model to perform well not only on head classes but also to effectively recognize data from tail classes. In this context, traditional intelligent fault diagnosis models often suffer from the issue of misclassifying tail-class samples as head-class samples. However, rare fault types in tail classes usually pose a greater threat to the system, making them more valuable in fault diagnosis.

    Self-supervised learning has become a mainstream approach to addressing the long-tailed distribution problem. Its training process consists of two stages: self-supervised pre-training and fine-tuning. In the self-supervised pre-training stage, the model’s feature layers are trained using an unlabeled dataset and a proxy task. In the fine-tuning stage, a linear classification layer is attached to the feature layers and trained on a labeled dataset to build the final classification model.
This thesis aims to study self-supervised learning-based fault diagnosis methods for long-tailed distribution data. The main contents include:
(1) Data Augmentation Strategy Search Algorithm: Since directly applying a simple Siamese network-based contrastive learning method to the fault diagnosis domain does not achieve the expected performance, we theoretically analyze the essential characteristics required for the key data augmentation module in a simple Siamese network. Based on this, we propose a data augmentation strategy search algorithm using a covariance matrix adaptation evolution strategy (CMA-ES). This algorithm iteratively searches for the optimal augmentation strategy parameters on a small-scale dataset, reducing the time cost associated with manually designing augmentation modules. The encoder trained with this algorithm demonstrates superior feature extraction capabilities compared to state-of-the-art (SOTA) methods such as SimCLR and BYOL. Additionally, through ablation studies, we challenge the claim in SimCLR’s original paper that “a larger batch size benefits contrastive learning models.”
(2) Semi-Supervised Decision Boundary Calibration: The simple Siamese network is more susceptible to the long-tail effect when fine-tuning the classifier layer, leading to lower classification accuracy than SimCLR. To mitigate this issue, we introduce the MARC decision boundary calibration algorithm. However, the MARC algorithm does not fully leverage the information from large-scale unlabeled datasets. To address this, we incorporate semi-supervised learning, where the model's predictions on unlabeled data are used as pseudo-labels. These pseudo-labeled samples are then combined with the original labeled dataset to fine-tune and calibrate the network.
Experimental results demonstrate that our proposed method outperforms existing SOTA methods, including SimCLR and BYOL, on a constructed long-tailed bearing fault diagnosis dataset, achieving superior performance.

\englishkeyword{Fault Diagnosis, Long-Tail Learning, Self-Supervised Learning, Siamese Network}
\end{englishabstract}

\thesistableofcontents

\chapter{绪\hspace{6pt}论}

\section{研究工作的背景与意义}
智能故障诊断是预测性健康管理(Prognostics and Health Management, PHM)系统中的核心组成部分,旨在实现设备故障的高效检测与精准识别,从而保障系统的安全与稳定运行。PHM 技术在各类关键工业装备中得到广泛应用,如复杂机械系统和高精度运转设备,具有重要的工程实践价值。智能故障诊断的有效性直接关系到生产过程的安全性、连续性与可靠性,具有重要的现实意义,例如旋转机械是现代工业中的核心设备,长期在高温、疲劳、重载等复杂工况下运行,若发生故障,不仅可能引发严重事故,还会造成巨大的经济损失,甚至威胁人员安全。传传统的智能故障诊断方法主要依赖于信号处理技术进行特征提取,并结合机器学习算法实现故障类型的判别与分类。相关研究在特征工程与分类性能方面已取得了诸多进展。然而,面对异构且海量的数据时,基于人工设计的特征提取方法通常高度依赖领域知识,不仅开发周期较长,还对专家经验提出较高要求,限制了其在实际应用中的灵活性与可扩展性,而且人工提取的特征在某一特定工况下可能表现良好,但在其他工况或不同设备上可能失效,缺乏泛化能力。
在高维特征空间中,许多人工构造的特征可能存在冗余性或与目标任务无关,进而导致分类器性能波动,甚至引发过拟合问题,影响模型的泛化能力。特征提取的难题不仅增加了诊断系统的复杂性和实施成本,还影响了故障诊断模型的稳定性、可靠性和适用性。因此,如何自动化、智能化地提取具有鲁棒性和泛化能力的故障特征,成为智能故障诊断领域亟待解决的重要问题。

现实中的故障检测数据通常呈现长尾分布\citing{nanjing2023faultdetection},这是一种偏态分布,其中头部类包含大量正常数据,而尾部类则包含较少的故障数据。如何基于长尾分布的故障检测数据训练有效的故障检测模型,是一个具有现实意义的难点问题\citing{cumm2023faultdiagnosis}。长尾分布数据在现实世界中广泛存在,大规模数据集通常呈现出长尾分布的特性。长尾分布的学习是故障诊断领域中的一项常见挑战。例如,正常工作状态的数据通常远多于故障工作状态,或者常见故障状态的数据量远多于罕见故障状态。现实中的故障检测数据往往服从长尾分布,这种分布是一种典型的偏态分布,如图\ref{fig_long_tail}所示:大规模训练数据中,头部类别的样本数量占据主导地位,而尾部类别的样本数量相对稀少,并呈现逐渐递减的趋势。长尾分布的数据常伴随显著的样本不平衡问题,导致训练过程中模型倾向于关注头部类别,从而削弱了对尾部类别的识别性能。如图\ref{tsne_mean_distribution}所示,当样本分布均衡时,各类别在特征空间中具有清晰的区分边界,并占据较宽广的特征空间。然而,当数据分布失衡呈现长尾分布时(如图\ref{tsne_lt_distribution}所示),尾部类别的特征分布变得狭窄,并依附于头部类别附近,从而扭曲了特征空间结构,降低了类别间的多样性与区分性。在使用长尾数据训练神经网络时,模型性能容易受到头部类别的影响,导致尾部类别的表现较差。然而,对尾部类别样本的错分往往带来更大的实际损失,因此研究尾部类别样本具有重要的现实意义。如何基于长尾分布的故障检测数据有效训练故障检测模型,已成为一个具有重要价值的研究难题。
\begin{figure}[h]
    \includegraphics[width=10cm]{long_tail.pdf}
    \caption{长尾分布示意图}
    \label{fig_long_tail}
\end{figure}

长尾分布数据对现代深度学习框架提出了巨大的挑战。即使使用数据重采样、类平衡损失等专门技术,在极端类不平衡的情况下,模型性能仍然显著下降。因此,为了更有效地应对这一挑战,深入理解长尾学习对特征分布的影响至关重要。然而,与平衡数据的学习情景不同,长尾学习中的标签扮演着一个复杂而矛盾的角色,形成了标签价值的困境。一方面,有标签的监督学习通常比无监督学习能够训练出更准确的分类器,这突显了标签的积极作用;另一方面,长尾的标签分布会自然引入“标签偏差”,其中头部类别在很大程度上驱动决策边界的学习,导致对尾部类别的压制。这表明,标签既是推动学习的动力,也可能成为性能下降的原因,可以说是一把双刃剑。自监督学习的提出旨在减少对人工标注数据的依赖,使得即便在缺乏标注数据的情况下,网络仍能高效地进行训练。自监督学习在预训练阶段通过人工构造输入数据的代理任务使特征层网络学习到有效的特征表示,完全摆脱了对标签的依赖。只在对最终的分类器层进行微调过程中运用了标签,从而最大限度地减小“标签偏差”带来的影响。文献\cite{zhang2021federated}指出,额外的自监督预训练能够显著提升常规长尾学习算法的性能,并已在CIFAR-LT、ImageNet-LT等视觉识别长尾数据集上得到验证。尽管在故障诊断的长尾学习领域,半监督学习、集成学习、样本加权等方法已经取得一定进展,但应用自监督学习方法的研究仍相对较少。因此,应用自监督学习解决长尾分布数据故障诊断问题仍是一个具有重要意义的研究方向。
\begin{figure}[h]
    \subfloat[]{
        \label{tsne_mean_distribution}
        \includegraphics[width=6.77cm]{tsne_mean_distribution.png}
    }
    \subfloat[]{
        \label{tsne_lt_distribution}
        \includegraphics[width=6.77cm]{tsne_lt_distribution.png}
    }
    \caption{均匀/长尾分布数据的tsne图(a)样本均匀分布时的T-SNE分布图(b)尾部类“1”类样本个为30,头部类样本个数为180个时的T-SNE分布图}
    \label{long-tail result}
\end{figure}

% 目前,在视觉识别领域,长尾学习已提出多种前沿方法,如变分自编码器\citing{kingma2013auto}、生成对抗网络\citing{goodfellow2020generative}、CB(Class-Balanced)Loss\citing{cui2019class}、自监督学习\citing{yang2020rethinking}、以及MARC决策面调整算法\citing{wang2023margin}等。然而,在故障诊断的长尾学习领域,常见的方案包括半监督学习\citing{nanjing2023faultdetection}、集成学习\citing{吴亮2023基于多级学习的长尾分布下交通多目标检测}、样本加权\citing{cumm2023faultdiagnosis}等,相较而言,前沿的长尾学习方法在故障诊断中的应用较少。因此,提出一个结合前沿长尾学习方法的故障诊断模型具有重要的现实意义。
\FloatBarrier  % 阻止后续浮动体越过这条线
\section{故障诊断基本流程}
故障诊断的基本流程通常包括数据采集、信号预处理、特征选择、模型训练与优化、以及故障识别与诊断结果分析等步骤。在现代工业系统中,设备状态的监测依赖于多种传感器,如加速度传感器、声发射传感器、电流传感器和温度传感器等,以采集振动、声音、电流和温度等信号,从而提供反映设备健康状况的数据基础。由于采集到的原始信号往往受到噪声干扰,信号预处理成为数据分析的关键步骤,常见的方法包括滤波、归一化、去趋势以及信号分解,以提高数据质量并增强关键特征的可识别性。

在信号预处理的基础上,特征提取是故障诊断中的核心环节,其目的是从原始信号中提取能够反映设备运行状态的关键信息。根据信号分析方式的不同,特征可以分为时域、频域和时频域特征。时域特征如均值、均方根、偏度、峰值因子和峭度等,主要用于描述信号的统计特性;频域特征通过傅里叶变换提取主频分量、功率谱密度和边频带特征,以揭示设备的动态特性;时频域特征则通过小波变换、短时傅里叶变换或其他时频分析方法提取,分析信号的非平稳性和时变特征。由于部分手工提取的特征可能存在冗余或无关信息,特征选择成为优化故障诊断性能的重要手段,通常采用主成分分析、互信息分析或递归特征消除等方法筛选最具诊断价值的特征,本文构造了深度神经网络并采用自监督学习方法训练特征层网络提取深层特征。

在获得高质量的特征后,模型训练是智能故障诊断系统构建的核心环节。传统的机器学习方法,常见的分类方法包括支持向量机、随机森林、K近邻和决策树等,依赖于人工构建的特征,能够在特定工况下实现较高的分类精度。然而,随着深度学习的发展,卷积神经网络、循环神经网络、Transformer和图神经网络等深度学习模型展现出更强的特征自动提取和故障识别能力,特别是在面对复杂工况和海量数据时,其优势尤为明显。由于工业环境中的数据分布通常存在不均衡问题,模型训练过程中需要考虑数据重采样、类别平衡损失等策略,以降低类别不平衡对故障识别能力的影响。此外,超参数优化在提升模型泛化能力和稳定性方面也起着至关重要的作用。

在训练得到故障诊断模型后,测试数据的输入可实现自动故障识别与分类。常见的故障类型包括轴承故障(如滚动体故障、内圈故障和外圈故障)、齿轮故障(如齿面点蚀、断齿和偏磨)以及电机故障(如定子故障、转子故障)等。故障分类的准确性通常采用准确率、召回率、F1-score等指标进行评估,以确保模型的可靠性和实用性。最终,故障诊断的核心目标是为设备维护和运行决策提供有效支持。通过诊断结果分析,可以实现设备健康状态评估、故障预警、剩余寿命预测以及故障溯源分析等功能,从而优化维护策略,提高设备运行的安全性和可靠性。随着深度学习、迁移学习和自监督学习等技术的不断发展,智能故障诊断方法正在逐步突破传统方法的局限,未来研究将进一步探索如何在复杂工况下提升诊断模型的鲁棒性和适用性,以推动智能运维的发展。
\FloatBarrier  % 阻止后续浮动体越过这条线
\section{故障诊断的研究现状}
近年来,故障诊断的研究如火如荼。薛阳等人\citing{xue2022multimodal}基于CWRU滚动轴承故障数据提出了将多模态融合与卷积神经网络相结合,提取了时域、频域两个模态的特征,实现时频域双模态融合对故障类型的联合诊断,相较基于单时域和单频域的卷积神经网络故障诊断模型都有提升。郭文军等人\citing{郭文军2022EMD}构建了经验模态分解与自回归相融合的深度特诊提取方法,提升了隐层筛选故障特征的能力。在构造故障诊断模型时特征提取显得颇为重要,因此故障诊断技术的特征提取得到了国内外学者的广泛关注。传统的时域信号特征提取方法,如FFT,小波包分解,小波变换,经验模态分解都得到了较好的应用。近年来,随着神经网络的发展,基于机器学习的特征提取方法的研究也如火如荼,如基于粒子群优化的自适应多稳态欠阻尼随机共振方法\citing{杨波2022自适应随机共振在信号特征提取中的应用},堆叠自动编码器\citing{蒋爱国2018基于多模态堆叠自动编码器的感应电机故障诊断},多尺度核卷积神经网络\citing{李永刚2023一种基于},递归图编码将振动信号转为增强信号特征的二维图像\citing{张龙2023采用递归图编码技术与残差网络的滚动轴承故障诊断}。

\section{长尾学习的发展历程和研究现状}
国内外在处理长尾问题时通常从三个方面着手,如图\ref{longtail_solution}所示:一是样本层面,采取欠采样(如随机欠采样、NearMiss\citing{shen2016near}、ENN\citing{wilson1972asymptotic})或过采样(如随机过采样、SMOTE\citing{chawla2002smote}、变分自编码器\citing{kingma2013auto}、生成对抗网络\citing{goodfellow2020generative})的方法。然而,由于新生成的样本中可能掺杂噪声,仍可能导致模型出现误判;二是损失函数层面,使用类平衡的损失函数,如基于数据频率逆加权\citing{huang2016learning},OHEM(Online Hard Example Mining)、Focal Loss\citing{lin2017focal},以及最新提出的CB(Class-Balanced)Loss\citing{cui2019class};三是模型层面,如通过集成学习、半监督和自监督学习\citing{yang2020rethinking}、MARC决策面调整算法\citing{wang2023margin},采用少数类样本与等量多数类样本的组合进行模型训练,还有迁移学习\citing{liu2019large,yin2019feature}、度量学习\citing{you2018scalable,zhang2017range} 和元学习\citing{jamal2020rethinking,shu2019meta},也得到了探索。最近的研究还发现,解耦特征层与分类器的结构可以带来更好的长尾学习结果\citing{zhou2020bbn, kang2019decoupling}。
在长尾分布问题的研究上,国内外已有大量的探索。例如,Liu J等人\citing{liu2020deep}提出了一种基于特征云的数据增强方法,通过从方差较大的头部类中学习类内多样性,并将其转移到尾部类,从而改善尾部类的分类效果,该方法已成功应用于人脸识别领域。Yang Y等人\citing{yang2020rethinking}探讨了不平衡样本标签在模型训练中的价值,提出通过半监督学习方法利用原模型识别无标签数据,并通过伪标签扩充原数据集,同时验证了自监督预训练在长尾学习中的有效性,且证明样本维度越高,性能提升越显著。吴磊等人\citing{wu2023personalized}提出了一种针对长尾图像的个性化专家识别算法,基于残差网络构建多专家学习模型,并引入个性化学习、信息融合与增强模块。通过两阶段学习方法,不仅显著提升了整体识别精度,还特别增强了中尾部类别的识别效果。吴亮等人\citing{吴亮2023基于多级学习的长尾分布下交通多目标检测}提出了多级分组分类器,以提升尾部类性能,同时避免头部类性能损失,并设计了基于多头注意力机制的分组特征重融合模块,为多级分类器输入更精细的特征。此外,基于多级分类器,他们还提出了Logit联合调整方法,以缓解组间不平衡问题。Wang Y等人\citing{wang2023margin}介绍了适用于长尾视觉识别的MARC决策面调整算法,通过对模型输出的预测分数进行额外训练,仅需三行代码便能实现。Cui Y等人\citing{cui2019class}提出了CB(Class-Balanced)损失函数,通过乘以与“独特”样本数量相关的因子,缓解了类别不平衡问题。CB因子与超参数β有关,通过平滑1到最常见类别的不平衡因子1/n来实现。

\begin{figure}[h]
    \includegraphics[width=12cm]{longtail_solution.png}
    \caption{长尾学习常见方法}
    \label{longtail_solution}
\end{figure}

综上所述,文献\cite{yang2020rethinking}为自监督预训练在故障诊断长尾学习中的应用提供了良好的启示,但半监督学习方法在现实长尾学习模型中的适用性尚待验证。文献\cite{wu2023personalized,吴亮2023基于多级学习的长尾分布下交通多目标检测}介绍的集成学习方法有助于减弱样本不平衡效应,虽然是常规方法,但如果在训练过程中融合自监督预训练,性能可能会得到进一步提升。文献\cite{wang2023margin}提出的决策面调整算法适用于一般长尾学习模型,但其在故障诊断中的可行性仍需进一步探讨,且该算法对模型性能提升的理论性解释也尚不完善。文献\cite{cui2019class}提出的损失函数方法具有创新性,但超参数$\beta$的人工选择在实际应用中可能面临一定的挑战。以上研究多集中于视觉长尾学习领域,并且在CIFAR-LT、ImageNet-LT等视觉识别数据集上取得了良好效果,而在故障诊断长尾学习领域,前沿的长尾学习算法应用较为匮乏。因此,将前沿的长尾学习算法应用于故障诊断领域,具有重要的研究价值,本文将着重研究模型层面的自监督长尾学习方法,分别研究自监督预训练和微调阶段过程的优化。
\FloatBarrier  % 阻止后续浮动体越过这条线
\section{自监督学习的发展历程和研究现状}

2019年,MoCo\citing{He_2020_CVPR}的提出掀起了视觉自监督学习的热潮,随后SimCLR\citing{chen2020simple}、BYOL\citing{grill2020bootstrap}、SwAV\citing{caron2020unsupervised}等主流自监督学习算法相继问世,使得该领域呈现出百花齐放、百家争鸣的繁荣局面。2021年末,MAE\citing{he2022masked}的提出进一步促进了自监督学习的发展,并将其提升到了一个全新的高度。然而,这一成就的背后,自监督学习经历了长期的迭代与发展。

目前,国内外学者对自监督预训练方法展开了大量研究。例如,在计算机视觉领域,Yang等人\citing{yang2020rethinking}研究了不平衡样本标签信息在模型训练中的价值,提出了一种半监督学习方法,该方法通过利用原模型识别无标签数据并生成伪标签来扩充原数据集,并验证了自监督预训练的有效性。此外,该研究还发现,样本维度越高,性能提升越显著,这为自监督预训练在故障诊断长尾学习领域的应用提供了有力支持。Doersch等人\citing{doersch2015unsupervised}提出了基于局部图像块位置预测的自监督任务,随机抽取图像中的两个块,训练模型预测一个块相对于另一个块的位置。Zhang等人\citing{zhang2016colorful}构建了基于图像上色的自监督任务,将图像转换到CIE Lab颜色空间后,提取其中的L通道作为模型输入,引导模型预测对应的a和b通道颜色信息。Gidaris等人\citing{gidaris2018unsupervised}通过人为旋转图像不同角度,并将旋转角度作为监督信号训练模型,从而实现图像语义特征的自监督学习。文献\cite{noroozi2017representation}提出了一种新的表示学习方式,通过基于计数视觉原语的人工监督信号来进行训练,而无需任何人工标注。核心思想是利用图像变换与表示变换之间的等变关系。文献\cite{caron2018deep}提出了利用KMeans算法生成样本的伪标签,将其作为数据的真实标签训练网络,用网络新提取的特征重新生成KMeans伪标签,以此重复。

近年来,自监督预训练方法也逐步应用于故障诊断领域,并受到广泛关注。例如,Zhang等人\citing{zhang2022prior}提出了基于先验知识的自监督任务,利用卷积自编码器进行训练,在小样本学习任务中取得了良好表现。Senanayaka等人\citing{senanayaka2020toward}采用One-Class SVM输出的标签作为代理标签,使用自监督预训练的卷积神经网络进行故障特征提取。W. Zhang等人\citing{zhang2021federated}提出了一种基于信号块交换的自监督任务,即通过打乱时域信号块顺序构造“伪数据”,并训练模型区分原始数据和伪数据。实验结果表明,该方法在特征提取任务中表现优异。

综上所述,尽管文献\cite{doersch2015unsupervised, zhang2016colorful, gidaris2018unsupervised}等研究在计算机视觉领域取得了重要进展,但其方法在故障诊断任务中的直接应用仍面临一定困难。然而,这些研究的思路为故障诊断领域的自监督预训练任务设计提供了有益的启示。例如,文献\cite{senanayaka2020toward}提出的One-Class SVM生成的标签可以结合聚类算法,使其适用于多分类任务;文献\cite{zhang2021federated}提出的信号块交换方法与文献\cite{doersch2015unsupervised}提出的图像块位置预测任务在思路上具有一定相似性,值得探索其结合的可能性。此外,文献\cite{zhang2022prior, senanayaka2020toward, zhang2021federated}提出的自监督预训练方法在任务设计上仍具有较大的创新空间,且在长尾学习任务中的应用尚未得到充分探索。因此,故障诊断领域的自监督预训练仍处于发展初期,借鉴计算机视觉领域的前沿自监督任务,提出更具创新性和更高性能的自监督预训练任务,仍具有重要的研究价值和挑战。
\FloatBarrier  % 阻止后续浮动体越过这条线
\section{对比自监督学习和孪生网络的发展历程和研究现状}

自监督学习的核心目标是在无人工标注的情况下自动提取有效特征\citing{min2021cross}。近年来,对比自监督学习(contrastive self-supervised learning)作为无监督学习的重要分支之一,通过构造正负样本对,使模型能够学习更具判别性的特征表示。它逐渐成为研究热点,并在图像识别任务中取得了最先进的(SOTA)性能。许多相关算法相继被提出,例如动量对比学习(Momentum Contrast, MoCo)\citing{He_2020_CVPR}、对比预测编码(Contrastive Predictive Coding, CPC)\citing{oord2018representation}以及用于视觉表征学习的简单对比学习框架(SimCLR)\citing{chen2020simple}。

这些方法的核心思想是通过拉近相同样本的不同视图(正样本对)并推远不同样本(负样本对)来学习有意义的特征表示。对比学习方法通常依赖存储在记忆库(memory bank)中的大量负样本,以提高对比学习效果\citing{wu2018unsupervised}。此外,一些其他对比学习方法不依赖负样本,而是计算两个正样本对之间的相似性,例如SwAV(Swapping Assignments between multiple Views)\citing{caron2020unsupervised}、BYOL(Bootstrap Your Own Latent)\citing{grill2020bootstrap}以及SDCT(Self-supervised Deep Correlation Tracking)\citing{yuan2020self}。

在对比自监督学习中,孪生网络(Siamese Network)因其适用于比较不同样本的网络结构,具有独特优势。Laine和Aila\citing{laine2016temporal}提出了基于孪生网络的$\pi$-model,用于训练深度神经网络,该方法在半监督学习场景下依赖不同的正则化规则与数据增强策略。Zheng和Yang\citing{zheng2019unsupervised}提出了一种“记忆正则化”机制,其核心思想是让主模型自身(而非外部模块)学习域内知识,以实现无监督场景自适应。Zhao等人\citing{zhao2021deep}提出了基于相互学习(mutual learning)和知识蒸馏(knowledge distillation)的训练方法,以提高视觉目标跟踪任务的性能。

上述研究均在各自领域中引入孪生网络的思想,以解决特定问题。本文受到SimSiam\citing{chen2021exploring}的启发,SimSiam采用简化的孪生网络结构,并在计算机视觉(CV)任务中展现出卓越性能。基于此,本文提出了一种新颖的轴承故障诊断框架,该框架可适用于由不同神经网络层构建的多种模型。与其他主流方法相比,SimSiam 具有以下三大优势:
\begin{enumerate}[label={(\arabic*)}]
    \item 无需构造负样本对;
    \item 无需采用大批量训练;
    \item 无需使用动量编码器(momentum encoder)。
\end{enumerate}
由于 SimSiam 模型高度依赖数据增强模块提供的视图差异性与语义一致性,其在故障诊断领域的性能受到增强策略质量的显著影响,尤其在故障特征弱或数据分布复杂的场景下,表现出较大的性能波动。本文从理论上分析了SimSiam数据增强模块对训练的意义以及所需要的数据增强方法需要具备的特质,同时通过一种自动搜索最优增强参数的方法解决SimSiam需要经过人工挑选数据增强方法和设定数据增强模块的参数的问题。
\FloatBarrier  % 阻止后续浮动体越过这条线
\section{本文的主要贡献与创新}

本论文聚焦于面向长尾分布数据的自监督学习故障诊断,主要的创新点和贡献如下:

\begin{enumerate}[label={(\arabic*)}]
    \item 提出了基于协方差矩阵适应进化策略的数据增强策略搜索算法优化的简单暹罗孪生网络的对比学习自监督预训练故障诊断框架。
    \item 通过半监督学习优化了决策面调整算法从而减小微调过程受长尾分布的影响。
\end{enumerate}

\section{本论文的结构安排}

本文的章节结构安排如下:

第一章介绍了研究的背景与意义,分析了传统故障诊断方法在处理长尾分布数据时面临的挑战,并重点回顾了国内外在长尾学习算法领域的研究进展与现有局限性。

第二章阐述了长尾学习与自监督学习范式的基本概念,介绍了为提出简单孪生网络故障诊断框架的理论基础,并介绍了实验使用的数据集和研究对象。

第三章提出了基于简单孪生网络为骨干网络的自监督预训练故障诊断框架,详细描述了框架各模块的设计与实现,并从实验与理论两个层面验证了框架的有效性。通过准确率、t-SNE分布图以及对Batch Size等参数的敏感性分析,证明了该方法在性能和稳定性上优于现有SOTA的自监督学习方法。

第四章提出了基于半监督学习与决策面调整相结合的微调框架,旨在减轻简单孪生网络在微调阶段受长尾效应影响的问题。实验结果表明,该微调框架不仅提升了简单孪生网络的整体性能,而且在其他模型中同样具有较好的效果,展示了其良好的普适性。

第五章总结了全文的研究工作,并对未来可能的研究方向进行了展望。


\chapter{长尾学习及相关方法的理论基础与轴承故障数据集}
\section{自监督学习相关理论}
自监督学习是一种范式,利用“代理任务”(pretext)自监督学习旨在挖掘大规模无标签数据中所蕴含的潜在结构,通过构造辅助任务生成伪监督信号,进而引导网络学习具备迁移至下游任务能力的通用表征。如图\ref{self_supervise_procedure}所示,该方法通常包含两个阶段:首先进行自监督预训练以学习初始表征,然后通过微调阶段在特定下游任务上进一步优化模型性能。在自监督预训练阶段,完全忽略数据的标签信息,专注于数据本身的特征学习。通过设计合理的自监督任务,引导模型从不平衡数据中提取与标签无关的潜在特征表示,以此减少类别偏差对模型初始化的影响,同时增强模型的泛化能力。在微调阶段,使用自监督预训练得到的网络权重作为初始化,并结合任务相关的监督信息进行优化,以提升模型在特定任务上的表现,同时保留预训练阶段学到的通用特征,从而提高模型的稳健性和泛化能力。在此基础上,可结合任意标准的不平衡学习技术进行后续训练,以学习最终的分类模型。由于预训练阶段和正常训练阶段是独立的,自监督学习可以与现有的不平衡学习方法无缝结合,增强模型性能,且自监督预训练阶段不依赖标签,使得网络能学习到更通用、更鲁棒的特征表示,从而避免了类别不平衡对特征学习的负面影响。
\begin{figure}[h]
    \includegraphics[width=10cm]{self_supervise_procedure.pdf}
    \caption{自监督训练范式流程}
    \label{self_supervise_procedure}
\end{figure}
\FloatBarrier  % 阻止后续浮动体越过这条线
\section{孪生网络与对比学习相关理论}
对比学习是一种通过比较正负样本对来提取有意义特征的学习方法。其基本假设是在学习到的特征嵌入空间中,相似样本应聚集在一起,而不相似样本应远离彼此。通过将学习任务视为辨别任务,对比学习能够帮助模型识别数据中的潜在特征和相似性。对比学习可分为监督对比学习和自监督对比学习两类。

监督对比学习是对比学习的一个子领域,依赖于标注数据来明确区分相似与不相似的样本。在监督对比学习中,模型通过训练数据对及其对应标签来学习哪些数据点是相似的,哪些是不相似的。其目标是学习一个表示空间,其中具备相似特征的实例聚集在一起,而特征不相似的实例则被推开。一种常用的优化目标是信息噪声对比估计(InfoNCE)损失函数,它通过最大化正样本对的相似性并最小化负样本对的相似性来优化模型。通过优化该目标,模型能够有效区分正负样本,进而提升下游任务的性能。

自监督对比学习则不同于监督学习,它从未标注的数据中学习深层且有效的特征表示。自监督对比学习通过设计代理任务,从未标注数据中生成正负样本对。这些代理任务旨在促使模型捕获数据中有意义的特征和相似性。自监督对比学习中常见的策略之一是利用数据增强技术构造同一样本的多个不同视图,并将这些视图作为正样本对,而来自不同样本生成的实例则作为负样本对。通过训练模型识别并区分正负样本对,模型能够学习到更丰富的语义信息,并在下游任务中取得较好的推广效果。

\section{t-SNE降维原理}
t-SNE(t-分布随机邻域嵌入)是一种广泛应用的非线性降维方法,常用于将高维数据映射到二维或三维空间,便于通过散点图直观地展示数据分布和聚类情况。它通常用于处理如图像、文本或高维特征数据的可视化任务。t-SNE 通过以下三个步骤来进行降维:

\begin{enumerate}[label={(\arabic*)}]
    \item \textbf{计算高维相似度}:t-SNE 首先计算每对数据点之间的相似度。在高维空间中,对于任意两点 \(x_i\) 和 \(x_j\),其相似度通过条件概率 \(p_{ij}\) 表示,计算公式为:
    
    \begin{equation}
    p_{ij} = \frac{\exp\left( -\frac{\|x_i - x_j\|^2}{2\sigma_i^2} \right)}{\sum_{j \neq i} \exp\left( -\frac{\|x_i - x_j\|^2}{2\sigma_i^2} \right)}
    \end{equation}
    
    其中,\(\|x_i - x_j\|\) 表示两点之间的欧氏距离,\(\sigma_i\) 是与点 \(x_i\) 相关的局部尺度参数,反映了该点的邻域大小。通过这种方式,t-SNE 保证了高维空间中相邻的数据点具有较高的相似度。
    
    \item \textbf{映射到低维空间}:接着,t-SNE 将数据映射到二维或三维的低维空间中。在低维空间中,t-SNE 使用 t-分布(Student's t-distribution)来计算数据点之间的相似度,公式为:    
    \begin{equation}
    q_{ij} = \frac{\left( 1 + \|y_i - y_j\|^2 \right)^{-1}}{\sum_{j \neq i} \left( 1 + \|y_i - y_j\|^2 \right)^{-1}}
    \end{equation}    

    其中,\(y_i\) 和 \(y_j\) 是低维空间中的数据点的向量,\(\|y_i - y_j\|\) 是它们之间的欧氏距离。t-分布的重尾特性使得它能够更加有效地处理聚类之间的距离,同时避免了在低维空间中过度拥挤的现象。
    
    \item \textbf{最小化KL散度}:t-SNE 的目标是最小化高维空间和低维空间中数据点相似度分布的差异。为此,t-SNE 通过最小化 Kullback-Leibler(KL)散度来优化模型的映射过程,KL 散度的计算公式为:    
    \begin{equation}
    \text{KL}(P || Q) = \sum_{i,j} p_{ij} \log \frac{p_{ij}}{q_{ij}}
    \end{equation}
    
    通过迭代优化,t-SNE 会调整低维空间中的数据点位置,直到高维和低维空间的相似度分布尽可能接近。
\end{enumerate}
t-SNE降维伪代码见表\ref{table:tsne_code}。
\begin{table}[h]
    \caption{t-SNE算法伪代码}
    \begin{tabular}{@{}l@{}} % 使用 @{} 去掉默认的左右边距,l 表示左对齐
    \toprule
    \multicolumn{1}{@{}l@{}}{\textbf{t-SNE算法伪代码}} \\ % 左对齐文本
    \midrule
    \begin{lstlisting}[basicstyle=\ttfamily,frame=none]
# 输入: 高维数据 X = {x1, x2, ..., xn}
# 输出: 低维数据 Y = {y1, y2, ..., yn}
# Step 1: 计算高维空间中数据点之间的相似度 p_ij
for i = 1 to n:  # 对每个数据点 xi
    for j = 1 to n:
        # 计算 xi 和 xj 之间的欧氏距离,并转化为条件概率 p_ij
        dist_ij = norm(xi - xj)
        p_ij = exp(-dist_ij^2 / (2 * sigma_i^2)) / \
               sum(exp(-dist_ij^2 / (2 * sigma_i^2)) for all j)
        P[i][j] = p_ij  # 存储 p_ij    
# Step 2: 初始化低维空间中的数据点 Y
Y = random_initialization(n, d)  # 随机初始化低维空间中的数据点
# Step 3: 迭代优化:最小化KL散度
for t = 1 to T:  # 最大迭代次数 T
    # 计算低维空间中的相似度 q_ij
    for i = 1 to n:
        for j = 1 to n:
            dist_ij = norm(yi - yj)
            q_ij = (1 + dist_ij^2)^(-1) / \
                   sum((1 + dist_ij^2)^(-1) for all j)
            Q[i][j] = q_ij  # 存储 q_ij

    # 计算KL散度并计算梯度
    KL = sum(P[i][j] * log(P[i][j] / Q[i][j]) for all i and j)
    gradients = compute_gradients(P, Q, Y)  # 计算梯度    
    # Step 4: 更新低维空间中的数据点 Y
    Y = Y - learning_rate * gradients  # 梯度下降更新低维数据点
    # Step 5: 终止条件判断
    if converged(KL):  # 如果KL散度收敛
        break
return Y  # 返回降维后的低维数据 Y
    \end{lstlisting} \\
    \bottomrule
    \end{tabular}
    \label{table:tsne_code}
\end{table}
\FloatBarrier  % 阻止后续浮动体越过这条线
\section{K-Means聚类原理}
K-Means 是一种经典的无监督学习算法,其核心思想是通过最小化样本点与各自簇中心之间的距离,将数据划分为 K 个簇。该算法在图像分割、客户细分、文档聚类以及特征提取等多个领域被广泛应用。算法步骤如下:
\begin{enumerate}[label={(\arabic*)}]
    \item \textbf{初始化}:从数据集中随机选取 \(K\) 个样本作为初始簇中心。
    \item \textbf{簇分配}:依据最小欧氏距离准则,将每个样本分配至其最近的簇中心所属的簇,欧氏距离计算为:
    \[
    \text{Distance}(x_i, c_j) = \| x_i - c_j \|
    \]
    其中,\(x_i\) 为数据点,\(c_j\) 为簇的质心。
    \item \textbf{更新质心}:重新计算每个簇的质心,即簇内所有数据点的均值:
    \[
    c_j = \frac{1}{|S_j|} \sum_{x_i \in S_j} x_i
    \]
    其中,\(S_j\) 为第 \(j\) 个簇的数据点集合,\(|S_j|\) 为该簇的数据点数。
    \item \textbf{重复迭代}:重复簇分配与簇质心更新步骤,直至簇质心收敛或达到预设的最大迭代次数。 
    \item \textbf{使用匈牙利算法找到最佳匹配}:在完成迭代后,使用匈牙利算法来优化簇标签的匹配,以确保最合适的标签分配。匈牙利算法可以通过最小化簇间的匹配成本来实现最佳匹配。
\end{enumerate}
KMeans算法的伪代码见表\ref{table:kmeans_code}。
\begin{table}[h]
    \caption{KMeans算法伪代码}
    \begin{tabular}{@{}l@{}} % 使用 @{} 去掉默认的左右边距,l 表示左对齐
    \toprule
    \multicolumn{1}{@{}l@{}}{\textbf{KMeans算法伪代码}} \\ % 左对齐文本
    \midrule
    \begin{lstlisting}[basicstyle=\ttfamily,frame=none]
# 输入: 数据集 X = {x1, x2, ..., xn}, 簇数 K, 最大迭代次数 T
# 输出: 簇标签 {y1, y2, ..., yn}

# 初始化:随机选择 K 个数据点作为初始质心 C = {c1, c2, ..., cK}
for t = 1 to T:  # 最大迭代次数 T
    for i = 1 to n:  # 对每个数据点 xi
        # 计算每个质心的距离,分配数据点给最近的质心
        distances = []  # 初始化一个空列表
        for cj in C:  # 对每个质心 cj
            dist = norm(xi - cj)  # 计算 xi 到 cj 的距离
            distances.append(dist)  # 添加到距离列表
        yi = argmin(distances)  # 分配数据点 xi 到最近的质心
        # 更新数据点的簇标签
        labels[i] = yi
    
    for j = 1 to K:  # 对每个簇
        # 更新质心为簇中所有数据点的均值
        cluster_points = 
            [xi for xi, label in zip(X, labels) if label == j]
        # 计算簇内数据点的均值作为新质心
        C[j] = mean(cluster_points)
    
    # 使用匈牙利算法找到最佳匹配
    optimal_labels = hungarian_algorithm(labels, C)
    
    # 如果质心不再变化
    if no_change_in_centroids(C):  # 如果质心不再变化
        break  # 退出

return optimal_labels  # 返回优化后的簇标签 {y1, y2, ..., yn}
    \end{lstlisting} \\
    \bottomrule
    \end{tabular}
    \label{table:kmeans_code}
\end{table}
\FloatBarrier  % 阻止后续浮动体越过这条线
\section{协方差矩阵适应进化策略原理}
协方差矩阵适应进化策略(CMA-ES)是一种进化算法,专用于解决连续优化问题。它是一种基于种群的优化方法,通过不断调整种群的协方差矩阵来引导搜索过程。CMA-ES 在解决非线性、非凸、高维优化问题时表现出色,广泛应用于机器学习、工程优化和科学研究中。CMA-ES 的更新规则如下:

\begin{enumerate}[label={(\arabic*)}]
 \item 采样新解:
\begin{equation}
\mathbf{x}_k^{(g+1)} = \mathbf{m}^{(g)} + \sigma^{(g)} \cdot \mathcal{N}(0, \mathbf{C}^{(g)})
\label{eq:sample}
\end{equation}
其中:
    \(\mathbf{x}_k^{(g+1)}\) 是第 \(g+1\) 代中的第 \(k\) 个候选解,
    \(\mathbf{m}^{(g)}\) 是第 \(g\) 代的均值向量,
    \(\sigma^{(g)}\) 是第 \(g\) 代的步长,
    \(\mathbf{C}^{(g)}\) 是第 \(g\) 代的协方差矩阵,
    \(\mathcal{N}(0, \mathbf{C}^{(g)})\) 是从多元正态分布中采样的随机向量。

 \item 更新均值:
\begin{equation}
\mathbf{m}^{(g+1)} = \sum_{i=1}^{\mu} w_i \mathbf{x}_{i:\lambda}^{(g+1)}
\label{eq:mean_update}
\end{equation}
其中:
    \(\mu\) 是选择的父代数量,
    \(w_i\) 是权重系数,
    \(\mathbf{x}_{i:\lambda}^{(g+1)}\) 是第 \(g+1\) 代中适应度排名前 \(\mu\) 的候选解。

 \item 更新协方差矩阵:
\begin{equation}
\mathbf{C}^{(g+1)} = (1 - c_1 - c_\mu) \mathbf{C}^{(g)} + c_1 \mathbf{p}_c^{(g+1)} (\mathbf{p}_c^{(g+1)})^\top + c_\mu \sum_{i=1}^{\mu} w_i \mathbf{y}_{i:\lambda}^{(g+1)} (\mathbf{y}_{i:\lambda}^{(g+1)})^\top
\label{eq:covariance_update}
\end{equation}
其中:
    \(c_1\) 和 \(c_\mu\) 是学习率,
    \(\mathbf{p}_c^{(g+1)}\) 是进化路径,
    \(\mathbf{y}_{i:\lambda}^{(g+1)} = (\mathbf{x}_{i:\lambda}^{(g+1)} - \mathbf{m}^{(g)}) / \sigma^{(g)}\)。

 \item 更新步长:
\begin{equation}
\sigma^{(g+1)} = \sigma^{(g)} \exp\left(\frac{c_\sigma}{d_\sigma} \left(\frac{\|\mathbf{p}_\sigma^{(g+1)}\|}{\mathbb{E}[\|\mathcal{N}(0, \mathbf{I})\|]} - 1\right)\right)
\label{eq:stepsize_update}
\end{equation}
其中:
    \(c_\sigma\) 是步长学习率,
    \(d_\sigma\) 是阻尼系数,
    \(\mathbf{p}_\sigma^{(g+1)}\) 是步长进化路径,
    \(\mathbb{E}[\|\mathcal{N}(0, \mathbf{I})\|]\) 是标准正态分布向量的期望范数。
\end{enumerate}
\FloatBarrier  % 阻止后续浮动体越过这条线

\section{数据集介绍}
\subsection{凯斯西储大学 CWRU数据集}
所使用的数据集由凯斯西储大学(Case Western Reserve University)提供。图 \ref{cwru_device} 展示了该实验平台的结构配置,其主要包括一台 2 马力的电动机、扭矩编码器、测功机及若干控制电路。实验中使用的轴承为 6205-2RS JEM 型深沟球轴承,分别安装在电机的驱动端和风扇端位置。

轴承故障是通过单点电火花放电加工(Electro-discharge Machining)制造的。根据故障位置,故障类型可分为以下四种:  
外圈故障(Outer Raceway Fault, OF);  
内圈故障(Inner Raceway Fault, IF);  
滚动体故障(Roller Faults, RFs);  
正常状态(Normal Condition, NC)。  
轴承故障的严重程度可由故障直径(Fault Diameter)描述,故障直径指轴承或其他机械部件上出现的故障或损伤的直径尺寸。 故障直径通常用来描述故障的大小和程度,对于故障诊断和预测维护非常重要。故障分类见表\ref{tab:cwru_fault_types}。
\begin{table}[H]
    \centering
    \caption{CWRU 轴承数据集故障类型表,其中0为正常类}
    \renewcommand\arraystretch{1.2}
    \begin{tabular}{ccc}
        \toprule
        序号 & 损伤部位 & 损伤直径 (mm) \\
        \midrule
        0  & - & - \\        
        1  & 外圈OF & 0.007 \\
        2  & 外圈OF & 0.014 \\
        3  & 外圈OF & 0.021 \\
        4  & 内圈IF & 0.007 \\
        5  & 内圈IF & 0.014 \\
        6  & 内圈IF & 0.021 \\
        7  & 滚珠BF & 0.007 \\
        8  & 滚珠BF & 0.014 \\
        9  & 滚珠BF & 0.021 \\
        \bottomrule
    \end{tabular}
    \label{tab:cwru_fault_types}
\end{table}

本研究采集了驱动端轴承的数据,采样频率为 12 kHz,故障类型包括内圈故障、外圈故障和滚动体故障,故障直径分别为 0.007 英寸、0.014 英寸和 0.021 英寸。不同故障类型的波形如图 \ref{cwru_samples} 所示。有关该数据集的更多详细信息,可访问 CWRU 轴承数据中心网站\citing{loparo2012case}。  

\begin{figure}[h]
    \centering
    \includegraphics[width=6cm]{cwru_device.jpeg}
    \caption{CWRU轴承测试设备}
    \label{cwru_device}
\end{figure}

\begin{figure}
    \centering
    \subfloat[]{
        \includegraphics[width=0.28\linewidth]{class_0.png}
        \label{class_0}
    }
    \subfloat[]{
        \includegraphics[width=0.28\linewidth]{class_1.png}
        \label{class_1}
    }
    \subfloat[]{
        \includegraphics[width=0.28\linewidth]{class_2.png}
        \label{class_2}
    }
    \\ % 换行
    \subfloat[]{
        \includegraphics[width=0.28\linewidth]{class_3.png}
        \label{class_3}
    }
    \subfloat[]{
        \includegraphics[width=0.28\linewidth]{class_4.png}
        \label{class_4}
    }
    \subfloat[]{
        \includegraphics[width=0.28\linewidth]{class_5.png}
        \label{class_5}
    }
    \\ % 换行
    \subfloat[]{
        \includegraphics[width=0.28\linewidth]{class_6.png}
        \label{class_6}
    }
    \subfloat[]{
        \includegraphics[width=0.28\linewidth]{class_7.png}
        \label{class_7}
    }
    \subfloat[]{
        \includegraphics[width=0.28\linewidth]{class_8.png}
        \label{class_8}
    }
    \\ % 换行
    \subfloat[]{
        \includegraphics[width=0.28\linewidth]{class_9.png}
        \label{class_9}
    }

    \caption{每个类别的样本信号示例:\textbf{(a)} NC;\textbf{(b)} IF(0.007);\textbf{(c)} BF(0.007)  \textbf{(d)} OF(0.007);\textbf{(e)} IF(0.014);\textbf{(f)} BF(0.014);\textbf{(g)} OF(0.014);\textbf{(h)} IF(0.021);\textbf{(i)} BF(0.021);\textbf{(j)} OF(0.021)}
    \label{cwru_samples}
\end{figure}
\FloatBarrier  % 阻止后续浮动体越过这条线
\subsection{帕德伯恩大学PU数据集}
PU轴承试验台由电机、测矩轴、滚动轴承试验模块、飞轮和负载电机组成,所有测试轴承均为6203型号滚动轴承。故障类别以不同损伤部位、损伤程度、损伤方法区分,加上正常类,共13类,具体故障类型如表\ref{tab:pu_fault_types}所示。
\begin{table}[H]
    \centering
    \caption{PU数据集故障类型分类表,其中0为正常类}
    \renewcommand\arraystretch{1.2}
    \begin{tabular}{ccccccc}
        \toprule
        序号 & 轴承编码 & 制造商 & 损伤程度 & 损伤部位 & 损伤方法 \\
        \midrule
        0  & K001 & IBU & - & - & - \\
        1  & KA01 & MTK & 1 & 外圈 & 电火花加工 \\
        2  & KA03 & LBU & 2 & 外圈 & 电雕刻 \\
        3  & KA05 & LBU & 1 & 外圈 & 电雕刻 \\
        4  & KA06 & LBU & 2 & 外圈 & 电雕刻 \\
        5  & KA07 & LBU & 1 & 外圈 & 转孔 \\
        6  & KA08 & LBU & 2 & 外圈 & 转孔 \\
        7  & KA09 & LBU & 2 & 外圈 & 转孔 \\
        8  & KI01 & MTK & 1 & 内圈 & 电火花加工 \\
        9  & KI03 & LBU & 1 & 内圈 & 电雕刻 \\
        10 & KI05 & LBU & 1 & 内圈 & 电雕刻 \\
        11 & KI07 & LBU & 2 & 内圈 & 电雕刻 \\
        12 & KI08 & LBU & 2 & 内圈 & 电雕刻 \\
        \bottomrule
    \end{tabular}
    \label{tab:pu_fault_types}
\end{table}


\FloatBarrier  % 阻止后续浮动体越过这条线
\subsection{长尾数据集构造}
模拟构建长尾分布数据集流程如图\ref{pareto}所示。设定不平衡因子 \(\beta = x_{\text{max}} / x_{\text{min}}\) 为数据集中样本数量最多的类与样本数量最少的类的数量之比。帕累托分布的概率密度函数为 \(p(x) = \frac{\alpha x_{\text{min}}^{\alpha}}{x^{\alpha+1}}\),令$x_{\text{min}} = 1$,则\(p(x) = \frac{\alpha}{x^{\alpha+1}}\)。以不平衡因子构建服从帕累托分布的长尾分布数据集,不同不平衡因子的帕累托分布概率密度函数如图\ref{paerto_fig_beta}所示,可以看到当最大类与最小类样本数的比值 \(\beta\) 增大时,帕累托分布的形状参数 \(\alpha\) 也随之增大。这表明分布的头部类别占比提高,而尾部类别占比显著降低,长尾效应越发显著。下面将介绍形状参数 \(\alpha\) 的求解方法,以及各类别样本占比的计算步骤。

\begin{figure}[h]
    \centering
    \includegraphics[width=12cm]{pareto.pdf}
    \caption{基于帕累托分布构造长尾分布的轴承数据集流程图}
    \label{pareto}
\end{figure}

\begin{figure}[h]
    \centering
    \includegraphics[width=12cm]{paerto_fig_beta.png}
    \caption{不同不平衡因子$\beta$的帕累托分布概率密度图}
    \label{paerto_fig_beta}
\end{figure}

已知帕累托分布的累积分布函数为:
\begin{equation}
F(x) = 1 - x^{-\alpha}, \quad x > 1
\end{equation}

每类的概率定义为:
\begin{equation}
P(n \leq x < n+1) = F(n+1) - F(n) = n^{-\alpha} - (n+1)^{-\alpha}
\end{equation}

设最大类与最小类的样本数比值为 $\beta$,则有以下关系式:
\begin{equation}
\beta = \frac{P(1 \leq x < 2)}{P(n \leq x < n+1)} = \frac{1 - 2^{-\alpha}}{n^{-\alpha} - (n+1)^{-\alpha}}
\end{equation}

目标是
    根据给定的 $\beta$ 和类别总数 $n$,求解形状参数 $\alpha$。
    其次计算每类的样本占比 $P(n \leq x < n+1)$。
    最后将所有类别的概率归一化,使其和为 1。
具体步骤如下:
    
首先求解 $\alpha$,根据 $\beta$ 的定义,解以下非线性方程以确定 $\alpha$:
\begin{equation}
\beta = \frac{1 - 2^{-\alpha}}{n^{-\alpha} - (n+1)^{-\alpha}}
\end{equation}
该方程一般无解析解,可以通过数值方法(如 Newton-Raphson 或其他优化算法)求解。

其次计算每类的概率,每类的概率可以通过以下公式计算:
\begin{equation}
P(n \leq x < n+1) = n^{-\alpha} - (n+1)^{-\alpha}, \quad n = 1, 2, \dots, N
\end{equation}

最后概率归一化,将所有类别的概率归一化,计算归一化后的概率:
\begin{equation}
P_{\text{norm}}(n \leq x < n+1) = \frac{P(n \leq x < n+1)}{\sum_{k=1}^N P(k \leq x < k+1)}
\label{pareto_sample}
\end{equation}
其中 $N$ 为类别总数,归一化后各类别的样本占比之和为 1:
\begin{equation}
\sum_{n=1}^N P_{\text{norm}}(n \leq x < n+1) = 1
\end{equation}

\chapter{基于暹罗孪生网络对比学习的自监督学习故障诊断方法}
本章提出了一种基于暹罗孪生网络对比学习的自监督学习故障诊断框架,设计了一种新颖的数据增强策略搜索算法,并对数据增强模块在该框架中的必要性和设计方向进行了详细的理论分析和实验。

设计孪生网络的一个关键挑战是防止网络输出出现“坍塌”现象,即输出最终趋向于一个常数的平凡解。为了解决这一问题,本章首先介绍了简单暹罗孪生网络框架,并结合实验与理论分析,探讨了各个模块在防止网络“坍塌”方面的作用。通过一系列实验证明了框架的有效性和鲁棒性,特别是在实际应用中如何通过合理设计避免模型的退化问题。
\FloatBarrier  % 阻止后续浮动体越过这条线
\section{模型整体架构及其模块设计}
\subsection{简单暹罗孪生故障诊断网络整体结构}
提出了一种用于故障诊断的简单暹罗孪生网络(Simple Siamese, SimSiam),其结构如图 \ref{simsiam_net} 所示。模型的训练过程包括三个阶段。首先是对比学习阶段,在该阶段,所有未标记的振动样本经过归一化预处理,并采用协方差矩阵适应进化策略(CMA-ES)搜索最优的数据增强策略,对输入样本进行增强,生成两个不同视图作为正样本对。然后,将正样本对输入模型进行SimSiam网络对比学习,以优化编码器的特征表示能力。接下来的微调阶段,提取第一阶段训练好的编码器,并在其最后一层附加一个线性分类层,随后使用标记样本对模型进行有监督训练。此外,线性分类层还可以被其他分类器替换,如KNN和SVM。最后,在故障诊断阶段,将编码器和线性分类器层组合成一个完整的故障分类器,应用于实际的故障诊断任务,实现高效的故障识别。以下将对各个模块的具体实现细节进行详细介绍。

\begin{figure}[h]
    \centering
    \includegraphics[width=16cm]{simsiam_net.pdf}
    \caption{简单暹罗孪生故障诊断网络}
    \label{simsiam_net}
\end{figure}
\FloatBarrier  % 阻止后续浮动体越过这条线
\subsection{简单暹罗孪生网络对比学习}
简单暹罗孪生网络(SimSiam——Simple Siamese)\citing{chen2021exploring}架构(图\ref{simsiam_arch})将信号$x$中的两个随机增强视图$x_1$和$x_2$作为输入。这两个视图由编码器(Encoder)网络$f$处理,该网络由骨干网络(例如ResNet\citing{he2016deep})和投影器MLP(Projector)组成。编码器$f$在两个视图之间共享权重。预测器MLP(Predictor)$h$转换一个视图的输出并将其与另一个视图进行匹配。将两个输出向量表示为$p_1 \triangleq h(f(x_1))$和$z_2 \triangleq f(x_2)$,最小化它们的负余弦相似度:
\begin{equation}
    \mathcal{D}(p_1, z_2) = -\frac{p_1}{\|p_1\|_2} \cdot \frac{z_2}{\|z_2\|_2}
    \label{eq:distance}
\end{equation}  
其中,$\|\cdot\|_2$ 是 $\ell_2$范数。定义对称损失函数
\begin{equation}
    \mathcal{L} = \frac{1}{2} \mathcal{D}(p_{1}, z_{2}) + \frac{1}{2} \mathcal{D}(p_{2}, z_{1})
\label{eq:cos_loss}
\end{equation}
该损失函数作用于单段信号,总损失值在计算所有信号的损失后取平均值。损失函数的最小值为-1。

方法中一个重要的组件是梯度停止(stop-grad)操作(图\ref{simsiam_arch})。通过修改公式(\ref{eq:distance}) 来实现它:
\begin{equation}
    D(p_1, \text{stopgrad}(z_2))
\label{eq:stopgrad}
\end{equation}
这意味着在这一项中,\( z_2 \) 被视为常数。类似地,公式(\ref{eq:cos_loss}) 的实现形式为:
\begin{equation}
\mathcal{L} = \frac{1}{2} D(p_1, \text{stopgrad}(z_2)) + \frac{1}{2} D(p_2, \text{stopgrad}(z_1))
\label{eq:cos_loss_stopgrad}
\end{equation}
第一项中 \( x_2 \) 的编码器不会从 \( z_2 \) 接收梯度,但在第二项中会从 \( p_2 \) 接收梯度(反之亦然,对于 \( x_1 \) 也是如此)。即在这一项中,\( z_2 \) 被视为常数。

简单暹罗孪生网络的伪代码如表\ref{table:simsiam_code}所示。

\begin{figure}[h]
    \includegraphics[width=12cm]{simsiam_arch.pdf}
    \caption{简单暹罗孪生网络}
    \label{simsiam_arch}
\end{figure}

\begin{table}
    \caption{简单暹罗孪生网络的伪代码,用Pytorch描述}
    \begin{tabular}{@{}l@{}} % 使用 @{} 去掉默认的左右边距,l 表示左对齐
    \toprule
    \multicolumn{1}{@{}l@{}}{\textbf{简单暹罗孪生网络Pytorch伪代码}} \\ % 左对齐文本
    \midrule
    \begin{lstlisting}[basicstyle=\ttfamily,frame=none]
#f: 骨干网络 + 投影器 MLP
#h: 预测器 MLP
for x in loader:  # 加载一个包含 n 个样本的小批量数据 x
    x1, x2 = aug(x), aug(x)  # 随机数据增强
    z1, z2 = f(x1), f(x2)  # 投影,形状为 n-by-d
    p1, p2 = h(z1), h(z2)  # 预测,形状为 n-by-d
    L = D(p1, z2)/2 + D(p2, z1)/2  # 损失函数

    L.backward()  # 反向传播
    update(f, h)  # SGD 更新

def D(p, z):  # 负余弦相似度
    z = z.detach()  # 停止梯度
    p = normalize(p, dim=1)  # 对 p 进行 L2 归一化
    z = normalize(z, dim=1)  # 对 z 进行 L2 归一化
    return -(p * z).sum(dim=1).mean()  # 计算负余弦相似度
    \end{lstlisting} \\
    \bottomrule
    \end{tabular}
    \label{table:simsiam_code}
\end{table}
引入停止梯度的设计暗示了另一个潜在的优化问题正在被隐式解决。SimSiam 是一种类似于期望最大化(EM)算法的方法,隐式涉及两组变量并解决两个潜在的子问题。停止梯度操作的引入是为了引入额外的变量集。

考虑以下形式的损失函数:
\begin{equation}
\label{eq:loss}
\mathcal{L}(\theta,\eta)=\mathbb{E}_{x,\mathcal{T}}\left[\|\mathcal{F}_{\theta} (\mathcal{T}(x))-\eta_{x}\|^{2}_{2}\right].
\end{equation}
其中,$\mathcal{F}$ 是由 $\theta$ 参数化的网络,$\mathcal{T}$ 是数据增强,$x$ 是输入样本。期望 $\mathbb{E}[\cdot]$ 是对输入样本和数据增强的分布进行的。为了便于分析,这里使用均方误差 $\|\cdot\|^{2}_{2}$,如果向量是 $\ell_{2}$-归一化的,则等效于余弦相似度。暂时不考虑预测器,稍后再讨论。

在式(\ref{eq:loss})中,引入了另一组变量,记为 $\eta$。$\eta$ 的大小与输入样本数量成正比。直观上,$\eta_{x}$ 是输入样本 $x$ 的表示,下标 ${}_{x}$ 表示使用输入样本索引访问 $\eta$ 的子向量。$\eta$ 不一定是网络的输出;它是一个优化问题的参数。

通过这种形式化,考虑解决以下问题:
\begin{equation}
\label{eq:min_loss}
\min_{\theta,\eta}\mathcal{L}(\theta,\eta).
\end{equation}
这里的问题是针对 $\theta$ 和 $\eta$ 的。这种形式化类似于 KMeans 聚类。变量 $\theta$ 类似于聚类中心:它是编码器的可学习参数。变量 $\eta_{x}$ 类似于样本 $x$ 的分配向量(在 KMeans 中是一个 one-hot 向量),它是 $x$ 的表示。

同样类似于 KMeans,式(\ref{eq:min_loss})中的问题可以通过交替算法解决,固定一组变量并解决另一组变量。形式上,可以在以下两个子问题之间交替:
\begin{equation}
\label{eq:theta_subproblem}
\theta^{t} \leftarrow \arg\min_{\theta}\;\mathcal{L}(\theta,\eta^{t-1})
\end{equation}
\begin{equation}
\label{eq:eta_subproblem}
\eta^{t} \leftarrow \arg\min_{\eta}\;\mathcal{L}(\theta^{t},\eta)
\end{equation}
其中,$t$ 是交替的索引,“$\leftarrow$”表示赋值。

可以使用随机梯度下降(SGD)来解决子问题(\ref{eq:theta_subproblem})。停止梯度操作是一个自然的结果,因为梯度不会反向传播到 $\eta^{t-1}$,而 $\eta^{t-1}$ 在这个子问题中是一个常数。

子问题 (\ref{eq:eta_subproblem}) 可以独立地为每个 $\eta_{x}$ 求解。现在的问题是最小化$\mathbb{E}_{\mathcal{T}}\left[\|\mathcal{F}_{\theta^{t}}(\mathcal{T}(x))-\eta_{x} \|_{2}^{2}\right]$,其中期望是对数据增强 $\mathcal{T}$ 的分布进行的。由于使用均方误差,可以通过以下方式轻松求解:

\begin{equation}
\label{eq:eta_update}
\eta_{x}^{t}\leftarrow\mathbb{E}_{\mathcal{T}}\Big{[}\mathcal{F}_{\theta^{t}}( \mathcal{T}(x))\Big{]}.
\end{equation}
这表明 $\eta_{x}$ 被赋值为 $x$ 在数据增强分布上的平均表示。

SimSiam 可以通过在式(\ref{eq:theta_subproblem})和式(\ref{eq:eta_subproblem})之间进行一次交替来近似。首先,通过仅对数据增强采样一次(记为 $\mathcal{T}^{\prime}$)并忽略 $\mathbb{E}_{\mathcal{T}}[\cdot]$ 来近似式(\ref{eq:eta_update}):
\begin{equation}
\label{eq:eta_approx}
\eta_{x}^{t}\leftarrow\mathcal{F}_{\theta^{t}}(\mathcal{T}^{\prime}(x)).
\end{equation}
将其代入子问题 (\ref{eq:theta_subproblem}),得到:

\begin{equation}
\label{eq:theta_update}
\theta^{t+1}\leftarrow\arg\min_{\theta}\mathbb{E}_{x,\mathcal{T}}\Big{[}\big{\| }\mathcal{F}_{\theta}(\mathcal{T}(x))-\mathcal{F}_{\theta^{t}}(\mathcal{T}^{ \prime}(x))\big{\|}_{2}^{2}\Big{]}.
\end{equation}
现在 $\theta^{t}$ 在这个子问题中是一个常数,而 $\mathcal{T}^{\prime}$ 由于其随机性暗示了另一个视图。这种形式化展示了孪生网络架构。其次,如果通过一步 SGD 减少损失来实现式(\ref{eq:theta_update}),那么可以接近 SimSiam 算法:一个自然带有停止梯度操作的孪生网络。

根据定义,预测器 $h$ 期望最小化:
\begin{equation}
\mathbb{E}_{x}\Big{[}\big{\|}h(z_{1})-z_{2}\big{\|}_{2}^{2}\Big{]} 
\label{eq:target_of_h}
\end{equation}
其中,$z_i = \mathcal{F}(x_i)$,$x_i$表示输入$x$的第$i$个视图。$h$ 的最优解应满足:对于任何输入$x$, $h(z_{1})\!=\!\mathbb{E}_{z}[z_{2}]\!=\!\mathbb{E}_{\mathcal{T}}\big{[}\mathcal{F}( \mathcal{T}(x))\big{]}$ 。这个项类似于式(\ref{eq:eta_update})中的项。在式(\ref{eq:eta_approx})的近似中,期望 $\mathbb{E}_{\mathcal{T}}[\cdot]$ 被忽略了。$h$ 的使用可能填补了这一空白。在实践中,实际计算期望 $\mathbb{E}_{\mathcal{T}}$ 是不现实的。但神经网络(例如预测器 $h$)可能能够学习预测期望,而 $\mathcal{T}$ 的采样隐式分布在多个 epoch 中。

根据上述讨论,$h$ 的最优解应满足 $h(z_1) = E_z[z_2]$。假设 $h$ 为线性映射,则有 $h(E[z_1]) = E[z_2]$。同时,根据对称损失函数式~(\ref{eq:cos_loss}),$h$ 的最优解应满足 $h(z_2) = E_z[z_1]$。将此关系代入,得到 $h(E[z_1]) = E[z_2]$,进一步推导得 $h(h(z_2)) = E[z_2]$。因此,可表示为:
\begin{equation}
    h(h(x)) = E[x]
    \label{eq:derivation_of_h}
\end{equation}

这表明,在满足最优解的条件下,$h$ 很可能是输入变量 $x$ 到其期望值的平滑映射。假设 $h = E$,显然这是式~(\ref{eq:derivation_of_h}) 的一个特解。基于此,子问题~(\ref{eq:theta_subproblem}) 可以表述为:

\begin{equation}
    \arg\min_{\theta} \left[\| \mathcal{F}_{\theta}(x_1) - E(z_2) \|_2^2 + \| \mathcal{F}_{\theta}(x_2) - E(z_1) \|_2^2 \right]
\end{equation}

因此,SimSiam 的目标是最小化两个视图 \( x_1 \) 和 \( x_2 \) 输出特征期望值之间的距离,即使得 \( E(\mathcal{F}_\theta(x_1)) = E[z_2] \) 且 \( E(\mathcal{F}_\theta(x_2)) = E[z_1] \),从而促使模型学习到在数据增强下保持不变的深层次语义。图~\ref{simsiam_derivation_of_features} 直观展示了模型训练前后特征分布的变化。在训练前,多个相似样本的两个视图的特征在特征空间中可能位于距离较远且形态各异的区域;而在训练后,两个视图的特征空间趋于重叠,且期望值之间的距离显著缩小。这表明,编码器成功学习到了对数据增强不敏感的深层次语义特征。此时,当 \( h \) 计算出 \( E[z_1] \) 时,即可得到式~(\ref{eq:target_of_h}) 的最优解,对于 \( z_2 \) 同理。

然而,如果数据增强模块没有有效地带来多样化的特征空间分布(例如,训练前两个视图的特征空间已经趋于重叠),那么当 \( h \) 学习到预测期望的能力时,编码器可能不再需要进一步学习那些与数据增强无关的深层语义特征。

\begin{figure}[h]
    \centering
    \includegraphics[width=12cm]{simsiam_derivation_of_features.pdf}
    \caption{不同视图的模型特征输出分布的变化示意图}
    \label{simsiam_derivation_of_features}
\end{figure}
\FloatBarrier  % 阻止后续浮动体越过这条线
\subsection{简单暹罗孪生网络的数据增强模块}
在对比表示学习算法中,数据增强(DA)起着至关重要的作用。网络的编码器(Encoder)能否成功地从振动信号中提取可区分的故障特征,依赖于为同一样本生成不同视图的质量。注意到,面向图像的最新(SOTA)对比表示学习算法广泛使用多种数据增强方法,而针对序列数据的增强方法则相对较少。
用于生成正样本对的数据增强方法需满足语义一致性的要求,即加入的噪声或变换不能改变样本的故障类别等核心语义特征。同时,增强方法还应具备足够的多样性,以在特征空间中引入有效的变化,从而促进编码器学习具有判别性的深层语义表示(依据上一节的理论分析)。基于这一原则,本文选取了九种在语义保持与特征多样性之间具有良好平衡的数据增强方法:

\begin{itemize}
    \item \textbf{掩码(Random Masked)}:用于遮盖输入数据的一部分,即随机挑选信号的某些部分用 0 替代,通常是在序列数据或图像数据中随机选择部分区域进行掩码处理。这有助于模型在训练时学会忽略一些无关信息,从而提高其鲁棒性。

    \item \textbf{添加高斯噪声(Adding Gaussian Noise)}:向数据中添加高斯噪声是一种常用的方法,旨在增强模型的鲁棒性。通过这种方式,模型能够在训练中适应噪声,从而提高其在噪声环境中的表现。

    \item \textbf{相位扰动(Phase Perturbation)}:修改信号的频率域中的相位信息,而保持幅度不变来生成新的数据样本。这种方法保留了信号的整体结构,但引入了细微的扰动,用于提高模型的泛化能力。

    \item \textbf{块打乱(RandomChunkShuffle)}:将时间序列数据分割成多个块,然后随机打乱这些块的顺序。这样可以创建不同的序列变体,增加模型对数据变异的适应能力,同时保持整体的语义不变。

    \item \textbf{随机缩放(Random Scaled)}:对数据进行随机幅度缩放来增强数据的方法。通过改变数据的尺度,模型能够学习到不同幅度下的数据模式,从而提升其泛化能力。

    \item \textbf{随机绝对值(Random Abs)}:对数据应用绝对值操作,将负值转换为正值或去掉负号。这个方法帮助模型处理包含负值的情形,增强其鲁棒性。

    \item \textbf{竖直翻转(Random Vertical Flip)}:随机地将数据进行竖直翻转来创建新的样本。这对于一些对竖直方向变化不敏感的任务非常有效。

    \item \textbf{水平翻转(Random Horizontal Flip)}:通过随机水平翻转图像或数据生成新样本,这对于具有较强对称性的数据特别有效,有助于提高模型的鲁棒性。

    \item \textbf{时移(Time Shift)}:将数据在时间轴上平移一定的时间步长来生成新的样本。这种方法可以帮助模型学会适应时间序列数据中事件的变化位置,提升其对时间依赖的理解能力。
\end{itemize}
同一输入信号分别经过上述数据增强方法后的视图如图\ref{data_augmentation}所示。
\begin{figure}[h]
    \centering
    \includegraphics[width=10cm]{data_augmentation.pdf}
    \caption{数据增强方法效果示意图}
    \label{data_augmentation}
\end{figure}

简单暹罗孪生网络(SimSiam)的一个显著问题是可能出现所有输出“坍塌”为常数的平凡解。SimSiam通过数据增强生成两个视图,并直接最大化同一输入的两个视图之间的相似性,而不依赖于负样本对。因此,其性能在很大程度上依赖于数据增强策略的质量(不改变原语义的同时丰富特征空间)。然而,人工设计数据增强策略不仅耗时耗力,还难以获得最优方案。如图 \ref{simsiam_no_cma_tsne} 所示,基于人工设定的数据增强策略(包括 4dB 强度的高斯噪声、50\% 概率的随机前后翻转,以及随机裁剪 100、200 或 300 个样本)训练的 SimSiam 网络在特征提取效果上存在明显不足。,该策略未能有效提取样本间的区分性特征,导致相对大一部分类别的样本特征“糅杂”在一起。该节提出了基于协方差矩阵适应进化策略(Covariance Matrix Adaptation Evolution Strategy,下称CMA-ES)的最优数据增强策略搜索算法,
将寻找最佳数据增强策略的问题形式化为一个搜索问题(见图 \ref{CMA-ES})。该方法由两个组件组成:搜索算法和搜索空间。在高层次上,搜索算法(实现为CMA-ES)采样一个数据增强策略 \( S \),该策略包含有关使用哪种图像处理操作、每批次中使用该操作的概率以及操作幅度的信息。此方法的关键在于,策略 \( S \) 将用于训练具有固定架构的神经网络,其验证准确率 \( R \) 将返回以评估适应度。
\begin{figure}[h]
    \centering
    \includegraphics[width=6cm]{simsiam_no_cma_tsne.png}
    \caption{基于人工设定数据增强策略(高斯噪声、随机前后翻转、随机裁剪)的SimSiam特征提取t-SNE图}
    \label{simsiam_no_cma_tsne}
\end{figure}
\begin{figure}[h]
    \centering
    \includegraphics[width=8cm]{CMA-ES.pdf}
    \caption{使用CMA-ES搜索最优的数据增强策略框架}
    \label{CMA-ES}
\end{figure}
搜索空间细节:在搜索空间中,一个策略由 8 个信号数据增强子策略组成,如图\ref{data_augmentation}所示。此外,每个操作还关联两个超参数:

\begin{enumerate}[label={(\arabic*)}]
    \item 应用操作的概率\(\in [0,1]\),
    \item 操作的幅度映射后\(\in [0,10]\)。
\end{enumerate}
映射为线性变换,如变量\(x \in [a,b]\)到\(y \in [0,10]\)的变换为
\begin{equation}
    y = \frac{10(x - a)}{b - a}
    \end{equation}
以下介绍各个数据增强子策略幅度的表示:
\begin{itemize}
    \item \textbf{掩码(Random Masked)}:随机选取100个不重叠的大小为$s$的子区间,将其置0。$s$作为增强幅度由\([1,5]\)映射为\([0,10]\)。

    \item \textbf{添加高斯噪声(Adding Gaussian Noise)}:将高斯噪声的信噪比SNR作为增强幅度,由\([2,6]\)映射为\([0,10]\)。

    \item \textbf{相位扰动(Phase Perturbation)}:相位扰动服从均匀分布$\text{U}(-perturb_{\text{max}}, perturb_{\text{max}})$,$perturb_{\text{max}}$作为增强幅度由\([0.1,0.5]\)映射为\([0,10]\)。

    \item \textbf{块打乱(RandomChunkShuffle)}:将时间序列数据分割成大小均匀的$s$个块,然后随机打乱这些块的顺序。$s$作为增强幅度由\([10,100]\)映射为\([0,10]\)。

    \item \textbf{随机缩放(Random Scaled)}:缩放的幅度服从均匀分布$\text{U}(1.0-scale, 1.0+scale)$。$scale$作为增强幅度由\([0.05,0.6]\)映射为\([0,10]\)。

    \item \textbf{随机绝对值(Random Abs)}:无幅度值。

    \item \textbf{竖直翻转(Random Vertical Flip)}:无幅度值。

    \item \textbf{水平翻转(Random Horizontal Flip)}:无幅度值。
    
\end{itemize}
则CMA-ES优化器求解的问题可以描述为 
\begin{equation}
    \arg\max_{\mathbf{p}, \mathbf{s}} \text{Accuracy}(\mathbf{p}, \mathbf{s}, \text{SimSiam-Net})
\end{equation}
其中,\(\mathbf{p}\) 和 \(\mathbf{s}\) 为求解的参数,分别满足 \(p_i \in [0, 1]\) 和 \(s_i \in [0, 10]\),Accuracy为简单暹罗网络验证的准确率。
\FloatBarrier  % 阻止后续浮动体越过这条线
\subsection{实验设置}
\textbf{编码器(Encoder),预测器(Predictor)和分类器(Classifier)}:编码器是该算法中故障诊断的关键部分。本研究主要使用其潜在编码来完成分类任务。由于简单暹罗网络是一个框架,可以根据需要构建编码器模型和预测器,其中预测器是一个相对简单的非线性函数。在第二阶段,需要在预训练的编码器后添加一个分类器层用于故障诊断,如图\ref{simsiam_net}所示。为了方便起见,本研究使用多个常规的卷积神经网络(CNN)块来构建编码器,并使用全连接层来构建预测器和分类器。每个卷积块(CB)包括四个层:一个 1-D 卷积层\( f_{\text{BN}} \)、一个激活函数层 \( f_{\text{ReLU}} \),以及一个池化层 \( f_{\text{Pool}} \)。 方程(\ref{eq:CNN})展示了 1-D 卷积层的操作,其中输入的形状为 \( (N, C_{\text{in}}, L) \),输出的形状为 \( (N, C_{\text{out}}, L_{\text{out}}) \),其中 \( N \) 是样本的数量,\( C \) 表示通道数,\( L \) 表示数据的长度。
\begin{equation}
    \begin{aligned}
    \operatorname{out}(N_i, C_{\mathrm{out}_j}) &= \operatorname{bias}(C_{\mathrm{out}_j}) + \sum_{k=0}^{C_{\mathrm{in}}-1} \operatorname{weight}(C_{\mathrm{out}_j}, k) \otimes \operatorname{input}(N_i, k).
    \end{aligned}
    \label{eq:CNN}
    \end{equation}
BN 层 \( f_{\text{BN}} \) 用于加速训练过程并减少由于层输入分布不同而导致的内部协变量偏移(ICS)的影响 \citing{ioffe2015batch},当模型较深时,这种影响尤为严重。公式(\ref{eq:BN})展示了批归一化的过程。
\begin{equation}
    f_{\text{BN}}(x_i, \mathbf{x}) = \gamma \frac{x_i - \mathbb{E}(\mathbf{x})}{\sqrt{\sigma(\mathbf{x})^2 + \epsilon}} + \beta
\label{eq:BN}
\end{equation}    
\begin{equation}
\mathbb{E}(\mathbf{x}) = \frac{1}{N}\sum_{i=1}^N x_i
\label{eq:exp_BN}
\end{equation}
\begin{equation}
\sigma(\mathbf{x}) = \sqrt{\frac{1}{N}\sum_{i=1}^N (x_i - \mathbb{E}(\mathbf{x}))^2}
\label{eq:sigma_BN}
\end{equation}
其中 \( x \) 是整个小批量(mini batch)的向量,批量大小为 \( N \),\( x_i \) 表示小批量中的第 \( i \) 个样本。  
\( \mathbb{E}(x) \) 和 \( \sigma(x) \) 分别表示向量批量的均值和方差,如公式 (\ref{eq:exp_BN}) 和 (\ref{eq:sigma_BN}) 所示。\( \epsilon \) 是一个超参数,通常设置为 \( 10^{-5} \),而 \( \gamma \) 和 \( \beta \) 分别是缩放因子和偏移因子,它们是可学习的参数,初始值分别设置为 1 和 0。
激活函数层 \( f_{\text{ReLU}} \) 可以表示为公式 (\ref{eq:relu}):
\begin{equation}
    f_{\text{ReLU}} = \max(0, x)
    \label{eq:relu}
\end{equation}
\( x_i^{(n)} \) 表示通过第 \( n \) 个卷积块(CB)的第 \( i \) 个输出向量,可以用公式 (\ref{eq:output_of_encoder}) 表示:
\begin{equation}
    \begin{aligned}
    x_i^{(n)} = f_{\text{Pool}}\bigl\{f_{\text{ReLU}}\bigl[f_{BN}\Bigl(f_{\text{conv}}\bigl(\mathbf{x}^{(n-1)}\bigr), f_{\text{conv}}\Bigl(x_i^{(n-1)}\Bigr)\Bigr)\bigr]\bigr\}
    \end{aligned}
    \label{eq:output_of_encoder}
\end{equation}

在该实验中,堆叠了十个 CNN 块作为编码器,其中卷积层的核大小设置为 3,填充设置为 1,步幅设置为 1,隐藏层采用 256 个核。使用两个线性全连接层构建预测器,分别具有 \( 256 \times 512 \) 和 \( 512 \times 256 \) 个神经元。在它们之间放置了一个批归一化层和一个修正线性单元(ReLU)层。分类器层部署了两个线性全连接层,分别具有 \( 256 \times 256 \) 和 \( 256 \times \text{num\_class} \) 个神经元,其中 \(\text{num\_class}\) 表示故障类别的数量。在两个线性层之间也放置了一个批归一化层和一个 ReLU 层。

\textbf{优化器(Optimizer)}:模型的训练不需要使用大批量优化器,例如分层自适应速率缩放(LARS)\citing{you2017large},因为所提出的方法可以在典型批量大小下工作,而不依赖于大批量训练。本研究使用 SGD 优化器训练模型。网络参数通过公式 (\ref{eq:SGD}) 更新。
\begin{equation}
    \theta_{l+1} = \theta_l - \eta_l \cdot \nabla_{\theta} \mathcal{L}(\mathbf{x}; \theta_l)
\label{eq:SGD}
\end{equation}
其中 \( \theta_t \) 表示时间 \( t \) 时的可学习参数,\( \eta_t \) 表示时间 \( t \) 时的学习率,\( L(\cdot) \) 表示损失函数。

学习率初始设置为 \( 0.05 \times \frac{\text{BatchSize}}{256} \),学习率采用余弦衰减调度。余弦衰减调度的公式为:
\begin{equation}
    \eta_t = \eta_{\text{min}} + \frac{1}{2} (\eta_{\text{max}} - \eta_{\text{min}}) \left(1 + \cos\left(\frac{t \pi}{T}\right)\right)
    \label{eq:cos_decay}
\end{equation}
    其中:
        \(\eta_t\) 是第 \(t\) 步的学习率,
        \(\eta_{\text{max}}\) 是初始学习率(最大学习率),
        \(\eta_{\text{min}}\) 是最小学习率,
        \(t\) 是当前训练步数,
        \(T\) 是总训练步数(衰减周期),
        \(\cos\) 是余弦函数。
    权重衰减为 0.0001,SGD 动量为 0.9。

\textbf{损失函数}:对比学习预训练阶段的损失函数为余弦相似度见式(\ref{eq:distance}),微调阶段的损失函数为交叉熵损失函数(Cross-Entropy Loss Function)。对于多分类问题,交叉熵损失函数表示为
\begin{equation}
    \mathcal{L}_{\text{CE}} = -\sum_{i=1}^{C} y_i \log(\hat{y}_i)
    \label{eq:cross_entropy}
    \end{equation}        
    其中:
        \( C \) 是类别数量,
        \( y_i \) 是真实标签的 one-hot 编码(第 \( i \) 类的真实概率),
        \( \hat{y}_i \) 是模型预测的第 \( i \) 类的概率。

\textbf{数据细节}:
根据不同不平衡因子构造帕累托分布的有标签数据集,测试集为均匀分布。每个样本包含 1024 个数据点。
    训练过程分为两个阶段:
    \begin{itemize}
        \item 在第一阶段,随机选择每个故障类别的 100 个无标签样本进行模型训练,训练总计 500 个周期。
        \item 在第二阶段,根据式(\ref{pareto_sample})构造服从帕累托分布的长尾分布有标签样本,以此微调模型,微调总计 150 个周期。
    \end{itemize}
    在整个训练过程中,每个输入模型的 mini-batch 大小设定为 \textbf{64}。
\textbf{微调(Fine-tuning)}:在对比学习阶段完成后,从模型中取出训练好的编码器 \( f \),并将其与一个多层感知器(MLP)模块连接,该MLP部署了两个线性全连接层,分别具有 \( 256 \times 256 \) 和 \( 256 \times \text{num\_class} \) 个神经元,其中 \(\text{num\_class}\) 表示故障类别的数量。在两个线性层之间也放置了一个批归一化层和一个 ReLU 层。用带标签的数据微调构建一个完整的分类器用于故障诊断任务,从而使模型获得分类能力。以准确率为性能指标,如公式 (\ref{eq:acc}):
    \begin{equation}
        \text{Accuracy} = \frac{TN + TP}{TN + TP + FP + FN}
        \label{eq:acc}
    \end{equation}
    其中,\( TN \) 表示真阴性,即正确预测为阴性的样本数;\( TP \) 表示真阳性,即正确预测为阳性的样本数;\( FP \) 表示假阳性,即错误分类为阳性的阴性样本数;\( FN \) 表示假阴性,即错误分类为阴性的阳性样本数。
    
    当验证集的样本均匀采样后(本研究采用此方法),准确率等于宏平均召回率(Macro-Averaged Recall)。宏平均召回率是对每个类别单独计算召回率,然后取平均值:
    \begin{equation}
        \text{Macro-Recall} = \frac{1}{C} \sum_{i=1}^{C} \frac{TP_i}{TP_i + FN_i}
        \label{eq:macro_recall}
    \end{equation}
    其中,\( C \) 是类别的总数,\( TP_i \) 和 \( FN_i \) 分别是第 \( i \) 类的真正例和假负例。
    
\textbf{性能评估}:  
    测试集由每个故障类别均匀地随机选取 \textbf{100} 个样本组成。需要注意的是,微调过程中使用的子集、对比学习过程中用到的无监督数据集与测试集完全独立。  
    用到的三种性能评估指标如下:

    \begin{itemize}
        \item 验证集准确率/宏平均召回率:在验证集上评估模型的分类准确率,以衡量模型的泛化能力。该指标表示了模型对未知数据的分类能力。计算公式如式(\ref{eq:acc})。宏平均召回率用于评估模型对各个类别的分类准确率,由于验证集的样本均匀分布,其数值上与准确率相等,计算公式如式(\ref{eq:macro_recall})。
        
        \item t-SNE特征提取可视化:使用t-SNE方法将高维特征降维至二维空间进行可视化分析,直观展示不同类别之间的聚类情况。该方法有助于分析模型对不同故障类别的区分能力。
        \item 特征标准差:如果输出 \( z \) 具有零均值各向同性高斯分布,可以证明 \( \frac{z}{\|z\|_2} \) 的标准差为 \( \frac{1}{\sqrt{d}} \)。下证:假设 \( z \in \mathbb{R}^d \) 是一个零均值各向同性高斯分布,即 \( z \sim \mathcal{N}(0, I_d) \),其中 \( I_d \) 是 \( d \) 维单位矩阵,表示协方差矩阵为单位矩阵,意味着每个分量 \( z_i \) 独立且服从标准正态分布。需要证明的是,\( \frac{z}{\|z\|_2} \) 的标准差(即每个分量的标准差)为 \( \frac{1}{\sqrt{d}} \),其中 \( \|z\|_2 = \sqrt{\sum_{i=1}^{d} z_i^2} \) 是 \( z \) 的 \( L_2 \) 范数。以下证明:

        设 \( z = (z_1, z_2, \dots, z_d) \),则 \( z \) 的每个分量 \( z_i \) 都是独立的标准正态随机变量 \( z_i \sim \mathcal{N}(0, 1) \),并且 \( \|z\|_2 = \sqrt{\sum_{i=1}^{d} z_i^2} \) 是 \( z \) 的欧几里得范数。关注向量 \( \frac{z}{\|z\|_2} \) 的每个分量。
        令 \( \hat{z} = \frac{z}{\|z\|_2} \),即 \( \hat{z} \) 是单位向量,表示将 \( z \) 投影到单位球面上的结果。
    
        因为 \( z \) 是各向同性高斯分布,其方向是均匀分布在单位球面上的。对于标准化后的向量 \( \hat{z} \),每个分量 \( \hat{z}_i = \frac{z_i}{\|z\|_2} \) 的分布不再是标准正态分布,而是受到范数约束的结果。由于 \( z \) 是各向同性的,\( \hat{z}_i \) 的分布是对称的,所有分量 \( \hat{z}_i \) 的期望值为 0。因此,接下来计算 \( \hat{z}_i \) 的方差。
        对于 \( \hat{z}_i \) 的方差,有:
        \begin{equation}
        \text{Var}(\hat{z}_i) = \mathbb{E}[\hat{z}_i^2] - (\mathbb{E}[\hat{z}_i])^2
        \end{equation}
        由于 \( \mathbb{E}[\hat{z}_i] = 0 \),只需要计算 \( \mathbb{E}[\hat{z}_i^2] \)。
    
        \begin{equation}
        \mathbb{E}[\hat{z}_i^2] = \mathbb{E}\left[\frac{z_i^2}{\|z\|_2^2}\right] = \frac{\mathbb{E}[z_i^2]}{\mathbb{E}[\|z\|_2^2]} = \frac{1}{d}
        \end{equation}
        其中 \( \mathbb{E}[z_i^2] = 1 \)(因为 \( z_i \) 是标准正态分布),而 \( \mathbb{E}[\|z\|_2^2] = \mathbb{E}[\sum_{i=1}^{d} z_i^2] = d \),因为每个 \( z_i^2 \) 的期望为 1。
        由于方差 \( \text{Var}(\hat{z}_i) = \frac{1}{d} \),因此标准差是:
        \begin{equation}
        \text{std}(\hat{z}_i) = \frac{1}{\sqrt{d}}
        \end{equation}

        \item 验证集KMeans分类准确率:在验证集上使用KMeans算法对Encoder提取的特征进行无监督分类,并计算分类准确率。通过与真实标签对比,评估模型在无监督情境下的表现,进一步验证其分类能力。
    \end{itemize}     


\textbf{故障诊断(Fault Diagnosis)}:在完成微调后,即可将模型应用于真实输入信号的故障诊断,以预测故障类别。
% \begin{figure}[h]
%     \centering
%     \includegraphics[width=14cm]{simsiam_fault_diag_procedure.pdf}
%     \caption{简单暹罗孪生故障诊断网络的训练流程图}
%     \label{simsiam_fault_diag_procedure}
% \end{figure}
\FloatBarrier  % 阻止后续浮动体越过这条线
\section{实验与分析}
为了验证所提出方法的有效性,本研究选择了CWRU数据集和PU数据集进行实验。为了验证所提方法的优越性,选择了几种流行的对比方法,分为传统机器学习算法和基于深度学习的智能方法两类。
\begin{enumerate}[label={(\arabic*)}]
    \item \textbf{SVM\citing{ziani2017bearing}}:多类 SVM 是一种强大且多功能的机器学习模型,能够处理线性或非线性的分类任务,通过构建多个二分类器来实现对多个类别的区分。首先,将计算原始数据的 16 个时域指标(均值、平方根、偏度等),并将它们作为 SVM 分类器的输入。

    \item \textbf{CNN\citing{eren2019generic}}:CNN 是一种监督学习方法。在这里,CNN 的网络架构与所提出方法的编码器 \(f\) 相同。

    \item \textbf{BYOL\citing{grill2020bootstrap}}:BYOL(Bootstrap Your Own Latent)是一种基于自监督学习的表示学习方法。与传统的对比学习方法不同,BYOL 不依赖于负样本对进行训练,而是通过最大化正样本对之间的相似度,从而学习到更具判别性的图像表示。该方法采用两个神经网络模块,其中一个作为在线网络,另一个作为目标网络,通过不断更新目标网络来提高模型的稳定性和性能。两个网络共享相同的编码器 \(f\),但参数更新的方式不同:在线网络的参数通过反向传播进行更新,而目标网络的参数则通过指数滑动平均进行更新。编码器 \(f\) 和数据增强方法与所提出方法相同。
    
    \item \textbf{SimCLR\citing{chen2020simple}}:SimCLR 是一种基于对比学习的自监督表示学习方法,通过构造正负样本对并最大化正样本对的相似度,从而学习出具有高质量表示的特征。该方法使用一个编码器 \(f\),通常是一个卷积神经网络(CNN),并通过数据增强技术生成不同的图像视图。SimCLR 的训练目标是最大化正样本对的相似性,同时最小化负样本对的相似性,来优化表示学习的质量。该方法的关键创新是引入了基于温度缩放的对比损失函数,进而提升了模型的表达能力。编码器 \(f\) 和数据增强方法与所提出方法相同。

\end{enumerate}
\FloatBarrier  % 阻止后续浮动体越过这条线
\subsection{数据增强策略搜索的结果}
\label{sec:CMA-ES_result}

采用CWRU 数据集 作为对比学习预训练的无标签数据,样本总数为 \textbf{500} 个。用于微调的有标签数据共 \textbf{100} 个,涵盖 \textbf{10} 个类别,每类包含 \textbf{10} 个样本。各类别信号的时域波形如图 \ref{cwru_samples} 所示。

在优化过程中,基于协方差矩阵适应进化策略(CMA-ES)进行最优数据增强策略的搜索。CMA-ES 的超参数设定如下:变量维度为 \textbf{14}(即 \(len(\mathbf{p}) + len(\mathbf{s})\)),种群规模设定为 \textbf{11},最大迭代次数设定为 \textbf{10} 轮。经过优化后,所获得的最优数据增强策略如表 \ref{CMA-ES_solution} 所示。

\begin{table}[h]
    \caption{CMA-ES 搜索最优的数据增强策略参数}
    \centering
    \begin{tabular}{cccc}
    \toprule
    增强编号 & 增强名称 & 概率 $p$ & 幅度 $s$\\
    \midrule
    DA0 & 高斯噪声 & 0.9761 & 9.6284 \\
    DA1 & 相位扰动 & 0.9899 & 3.4435 \\
    DA2 & 块打乱 & 0.9687 & 1.3436 \\
    DA3 & 掩码 & 0.7873 & 8.1961 \\
    DA4 & 缩放 & 0.5776 & 7.8206 \\
    DA5 & 绝对值 & 0.4020 & -- \\
    DA6 & 竖直翻转 & 0.4392 & -- \\
    DA7 & 水平翻转 & 0.4420 & -- \\
    \bottomrule
    \end{tabular}
    \label{CMA-ES_solution}
\end{table}
\FloatBarrier  % 阻止后续浮动体越过这条线
\subsection{在CWRU数据集和PU数据集上的实验结果}
\label{simsiam_cwru_results}
在本节中,采用表\ref{CMA-ES_solution} 所示的数据增强参数,并使用 CWRU 数据集和PU数据集作为数据集。对比学习预训练的无标签数据集的样本总数为 100$\times$类别数个,用于微调的有标签数据共计10$\times$类别数个,并通过帕累托分布生成不同不平衡因子的长尾分布标记样本。最终,在样本分布均匀的测试集上计算分类准确率,并进行 5 次重复试验取平均值。表\ref{simsiam_simclr_byol_cnn_svm_results} 和表\ref{pu_simsiam_simclr_byol_cnn_svm_results} 展示了在不同不平衡因子 $\beta$ 下,五种方法(SimSiam、SimCLR、BYOL、CNN 及 SVM)在 CWRU 数据集与 PU 数据集上的分类准确率与宏平均召回率。宏观上看,对比学习方法(尤其是 SimCLR 和 SimSiam)在不同程度的不平衡条件下均表现出较强的鲁棒性与稳定性,显著优于传统的浅层方法SVM及基于监督学习的 CNN。

在 CWRU 数据集上,SimCLR 在所有 $\beta$ 值下均取得了最优或接近最优的结果,平均准确率/召回率达到 0.9265,表现最为稳健。SimSiam 紧随其后,平均值为 0.9131,亦具备良好的不平衡适应能力。相较之下,BYOL 平均性能为 0.8610,略低于前两者,而 CNN 和 SVM 的平均值分别为 0.7420 与 0.4305,显示出其对不平衡分布较为敏感,尤其在 $\beta = 100$ 时性能大幅下降。

类似地,在 PU 数据集上,SimCLR 与 SimSiam 分别以 0.6494 和 0.6412 的平均值占据前两位,显著高于 BYOL(0.6001)、CNN(0.5211)与 SVM(0.2989)。尤其值得注意的是,随着不平衡程度的加剧($\beta$ 从 1 增至 100),SVM 表现急剧下滑,说明其对类别分布极为敏感,而对比学习方法仍能保持相对稳定的性能。

图\ref{tsne_of_all_models} 展示了 SimSiam、SimCLR、BYOL、CNN 和 SVM 五种方法(CNN和SVM基于不平衡因子$\beta$为1训练)在均匀分布与长尾分布数据集上的 t-SNE 特征投影。通过这些图示,可以直观地观察不同类别样本在低维空间中的分布情况,从而评估各方法的特征提取效果。SimSiam 的 t-SNE 可视化结果(图 \ref{simsiam_tsne})展示了较为理想的特征分布,不同类别之间边界清晰,且各类样本未出现明显混合,聚类效果良好。相比之下,其余模型的 t-SNE 图中普遍存在不同类别样本的交叉重叠现象,给线性分类器带来了较大的分类困难。然而,该可视化结果与表 \ref{simsiam_simclr_byol_cnn_svm_results} 中的定量评估结果存在一定差异,针对该问题将在下一章中进一步分析并提出改进策略。

\begin{table}[!h]
    \caption{CWRU数据集SimSiam、SimCLR、BYOL、CNN 与 SVM 在不同 $\beta$ 值下的准确率/宏平均召回率}
    \centering
    \begin{tabular}{ccccccc}
    \toprule
    不平衡因子$\beta$  & SimSiam & SimCLR & BYOL & CNN & SVM \\
    \midrule
    1   & 0.9208  & 0.9398 & 0.8904 & 0.8979 & 0.7198 \\
    10  & 0.9423  & 0.9458 & 0.8942 & 0.8538 & 0.2990 \\
    50  & 0.9133  & 0.9198 & 0.8481 & 0.6804 & 0.3917 \\
    100 & 0.8760  & 0.9004 & 0.8112 & 0.5358 & 0.3115 \\
    \midrule
    平均值 & \textbf{0.9131} & \textbf{0.9265} & 0.8610 & 0.7420 & 0.4305 \\
    \bottomrule
    \end{tabular}
    \label{simsiam_simclr_byol_cnn_svm_results}
\end{table}

\begin{table}[!h]
    \caption{PU数据集SimSiam、SimCLR、BYOL、CNN 与 SVM 在不同 $\beta$ 值下的准确率/宏平均召回率}
    \centering
    \begin{tabular}{ccccccc}
    \toprule
    不平衡因子$\beta$  & SimSiam & SimCLR & BYOL & CNN & SVM \\
    \midrule
    1   & 0.6443  & 0.6543 & 0.6246 & 0.6273 & 0.5036 \\
    10  & 0.6627  & 0.6597 & 0.6229 & 0.6018 & 0.2046 \\
    50  & 0.6418  & 0.6511 & 0.5870 & 0.4770 & 0.2710 \\
    100 & 0.6158  & 0.6324 & 0.5660 & 0.3784 & 0.2163 \\
    \midrule
    平均值 & \textbf{0.6412} & \textbf{0.6494} & 0.6001 & 0.5211 & 0.2989 \\
    \bottomrule
    \end{tabular}
    \label{pu_simsiam_simclr_byol_cnn_svm_results}
\end{table}

\begin{figure}[h]
    \centering
    \subfloat[]{
        \includegraphics[width=0.25\linewidth]{simsiam_tsne.png}
        \label{simsiam_tsne}
    }
    \subfloat[]{
        \includegraphics[width=0.25\linewidth]{simclr_batch_size_16.png}
        \label{simclr_tsne}
    }
    \subfloat[]{
        \includegraphics[width=0.25\linewidth]{byol_batch_size_128.png}
        \label{byol_tsne}
    }
    \\
    \subfloat[]{
        \includegraphics[width=0.25\linewidth]{cnn_tsne.png}
        \label{cnn_tsne}
    }
    \subfloat[]{
        \includegraphics[width=0.25\linewidth]{svm_tsne.png}
        \label{svm_tsne}
    }
    \caption{验证集数据经特征提取的 T-SNE 可视化:(a) SimSiam 的 T-SNE 分布图;(b) SimCLR 的 T-SNE 分布图;(c) BYOL 的 T-SNE 分布图;(d) CNN 的 T-SNE 分布图;(e) SVM 的 T-SNE 分布图}
    \label{tsne_of_all_models}
\end{figure}
\FloatBarrier  % 阻止后续浮动体越过这条线
\subsection{消融实验}
本小节基于CWRU数据集对SimSiam网络的停止梯度模块、预测器模块和数据增强模块的作用进行了实验分析,并对无标签数据集规模、Batch Size等训练关键参数进行了实验。

\textbf{停止梯度和预测期模块}

图\ref{train_process_simsiam_discuss}展示了“使用停止梯度(Stop-Grad)”与“不使用停止梯度”在CWRU数据集上的比较。架构和所有超参数保持不变,唯一的区别是是否使用停止梯度或预测器。
图\ref{train_process_simsiam_discuss:loss}显示了训练损失。如果不使用停止梯度,优化器可能会迅速找到一个退化解,并使损失函数达到最小值\(-1\),但这通常并不是一个有效的解,因为模型可能会过拟合或陷入局部最优解。为了证明这种退化是由坍缩引起的,以下研究了 \(\ell_2\) 归一化输出 \( z / \|z\|_2 \) 的标准差。如果输出坍缩到一个常数向量,则所有样本的标准差在每个通道上应该为零,这可以从图\ref{train_process_da_discuss:std}中的蓝色或绿色曲线中观察到。
作为对比,如果输出 \( z \) 具有零均值各向同性高斯分布, \( \frac{z}{\|z\|_2} \) 的标准差为 \( \frac{1}{\sqrt{d}} \),从图\ref{train_process_da_discuss:std}的红色曲线可以观察到。

图\ref{train_process_simsiam_discuss:std}的红色曲线显示,使用停止梯度时,标准差接近 \( \frac{1}{\sqrt{d}} \)。这表明输出没有坍缩,而是分散在单位超球面上。
图\ref{train_process_simsiam_discuss:acc}绘制了KMeans聚类的验证准确率。KMeans聚类可以作为进展的监控工具。使用停止梯度时,监控显示出准确率稳步提高。
图\ref{train_process_simsiam_discuss}的绿色曲线显示模型在移除预测器(Predictor)后的表现,可以看出移除预测器后发生了与移除Stop-Grad类似的坍塌。事实上,如果使用对称损失(式(\ref{eq:cos_loss})),这一现象是可以预期的。现在的损失函数是
\begin{equation}
    \frac{1}{2}D(z_1, \text{stopgrad}(z_2)) + \frac{1}{2}D(z_2, \text{stopgrad}(z_1)).
\end{equation}
其梯度的方向与 $D(z_1, z_2)$ 的梯度相同,只不过幅度被缩放了 $\frac{1}{2}$。在这种情况下,移除Stop-Grad相当于将损失函数缩放为$\frac{1}{2}$。这一调整导致了模型的崩溃(见图\ref{train_process_simsiam_discuss})。图\ref{tsne_simsiam_discuss}进一步可视化了该崩溃对特征提取的影响:在移除Stop-Grad或预测器后,特征空间中的大量样本交叉重叠,导致其变得难以区分。
表\ref{simsiam_vs_simsiamwosg_wopred}也显示了移除Stop-Grad或预测器后的巨大性能损失。实验表明,Stop-Grad模块对防止模型塌陷起到至关重要的作用,当移除该模块后,通过观察最小可能的损失值以及模型输出的常数性(标准差),可以发现这种塌陷现象。塌陷解的存在表明,仅通过架构设计(如预测器、BN、$\ell_{2}$-范数)来防止塌陷是不充分的。

\begin{figure}
    \centering
    \subfloat[]{
        \includegraphics[width=0.45\linewidth]{loss.png}
        \label{train_process_simsiam_discuss:loss}
    }
    \subfloat[]{
        \includegraphics[width=0.45\linewidth]{std.png}
        \label{train_process_simsiam_discuss:std}
    }\\
    \subfloat[]{
        \includegraphics[width=0.45\linewidth]{acc.png}
        \label{train_process_simsiam_discuss:acc}
    }
    \caption{SimSiam,SimSiam(无Stop-Grad模块)与SimSiam(无预测器模块)的训练过程对比(a) 损失函数;(b) 提取特征的标准差;(c) K-Means聚类准确率}
    \label{train_process_simsiam_discuss}
\end{figure}

\begin{figure}
    \centering
    \subfloat[]{
        \includegraphics[width=0.3\linewidth]{simsiam_tsne.png}
        \label{tsne_simsiam_discuss:simsiam_tsne}
    }
    \subfloat[]{
        \includegraphics[width=0.3\linewidth]{simsiam_wo_stopgrad_tsne.png}
        \label{tsne_simsiam_discuss:simsiam_wo_stopgrad_tsne}
    }
    \subfloat[]{
        \includegraphics[width=0.3\linewidth]{simsiam_wo_pred_tsne.png}
        \label{tsne_simsiam_discuss:simsiam_wo_pred_tsne}
    }
    \caption{SimSiam,SimSiam(无Stop-Grad模块)与SimSiam(无预测器模块)的T-SNE图(a) SimSiam;(b) SimSiam(无Stop-Grad模块);(c) SimSiam(无预测器模块)}
    \label{tsne_simsiam_discuss}
\end{figure}

\begin{table}
    \caption{SimSiam、SimSiamw/o SG 与 SimSiamw/o Pred 在不同 $\beta$ 值下的准确率/宏平均召回率}
    \centering
    \begin{tabular}{cccc}
    \toprule
    不平衡因子$\beta$  & SimSiam & SimSiamw/o SG & SimSiamw/o Pred \\
    \midrule
    1   & 0.9515  & 0.5338  & 0.5217 \\
    10  & 0.9402  & 0.4919  & 0.4604 \\
    50  & 0.9131  & 0.3579  & 0.3892 \\
    100 & 0.8450  & 0.3412  & 0.3704 \\
    \midrule
    平均值 & 0.9125 & 0.4312 & 0.4354\\
    \bottomrule
    \end{tabular}
    \label{simsiam_vs_simsiamwosg_wopred}
\end{table}

\textbf{数据增强模块}

\ref{sec:CMA-ES_result}小节在限定范围内求得了各个数据增强子策略的最优参数解,并在之后的实验中取得了良好的结果。以下通过实验分析表\ref{CMA-ES_solution}中的8个数据增强子策略对模型性能的影响。图\ref{train_process_da_discuss}显示了移除单个子策略后模型训练过程中的损失函数及特征标准差。表\ref{tb:da_discuss_results}移除不同数据增强子策略的准确率性能变化。以下分析了不同数据增强方法对模型性能的影响。具体而言,讨论了以下几种增强技术:缩放、块打乱和掩码、高斯噪声和相位扰动、绝对值以及竖直翻转和水平翻转。

首先,缩放操作对模型性能的影响最为显著。从表\ref{tb:da_discuss_results}的结果可以看出,移除缩放数据增强后,模型的性能显著下降。振动信号的幅值在时间轴上并非稳定,而是存在波动,这表明信号的特征在不同的尺度和幅度上有变化。引入缩放操作后,模型能够在较小的时间范围内提取出更大幅值范围的特征,从而增强了模型的泛化能力。缺少缩放操作后,模型在不同幅度范围内的鲁棒性降低,导致性能下降。通过增加数据的多样性,缩放操作帮助模型适应了不同的目标尺度和视野范围,进而提升了模型的表现。

其次,块打乱和掩码操作通过局部区域的扰动提高了模型的鲁棒性和对局部特征的提取能力。具体而言,块打乱操作通过打乱图像或信号的局部区域,迫使模型学习局部特征而非依赖全局信息,从而避免模型对局部区域的过度拟合。而掩码操作通过对部分区域进行遮挡,模拟数据中的缺失或遮挡情况,促使模型从有限的信息中恢复或推断出其余部分。这两种数据增强方法在特定场景下,尤其是在存在局部信息缺失或遮挡的情况下,能够有效提升模型的鲁棒性和泛化能力。

高斯噪声和相位扰动主要通过引入轻微的随机噪声或频域变化来增加数据的扰动。尽管这类增强方法能够增加数据的多样性,但生成的样本与原始数据的差异较小,可能无法充分捕捉到数据的多种变异情况,从而限制了模型的泛化能力,未能有效扩展模型的学习能力,因此对模型性能的影响较为有限。

绝对值操作通过将信号中的负值转化为正值,从而扩展了模型所接触到的样本空间。尤其是在处理具有周期性或对称特征的信号时,绝对值能够引入更多样化的训练样本。这种方法帮助模型适应不同的变化模式,并有效防止模型过早陷入“塌陷”现象。此外,绝对值操作增强了模型对信号反转的鲁棒性,有助于提升模型的泛化能力,特别是在面对多变输入数据时。

最后,竖直翻转和水平翻转操作在处理周期性信号时效果较为有限。周期性信号通常具有特定的模式或周期性变化,这些翻转操作改变了信号的空间结构,但对信号的本质特征影响较小。因此,这两种操作对模型性能的提升相对较弱,尤其是在周期性信号的情况下。

综上所述,不同的数据增强方法对模型性能的提升作用不同。缩放操作显著增加了数据的多样性,提升了模型的泛化能力,因此对模型性能的提升最为显著。其次,块打乱和掩码操作在局部信息缺失或遮挡的情况下有助于提高模型的鲁棒性。尽管高斯噪声和相位扰动对模型性能的提升作用较小,但它们仍然能够增加数据的多样性。最后,绝对值操作有效防止了模型的“塌陷”现象,增强了模型对信号反转的鲁棒性,而竖直翻转和水平翻转的作用较为有限,特别是在处理周期性信号时。

\begin{table}
    \caption{移除不同数据增强子策略在不同 $\beta$ 值下的准确率/宏平均召回率}
    \centering
    \begin{tabular}{ccccccccc}
    \toprule
    不平衡因子$\beta$  & DA0 & DA1 & DA2 & DA3 & DA4 & DA5 & DA6 & DA7 \\
    \midrule
    1   & 0.9279  & 0.9127 & 0.9254 & 0.9204 & 0.9106 & 0.9298 & 0.9394 & 0.9338  \\
    10  & 0.9258  & 0.9348 & 0.8944 & 0.9298 & 0.8792 & 0.9358 & 0.9333 & 0.9442  \\
    50  & 0.8563  & 0.8379 & 0.8221 & 0.8467 & 0.7490 & 0.8165 & 0.8573 & 0.8387  \\
    100 & 0.7935  & 0.8169 & 0.7356 & 0.7194 & 0.7119 & 0.7912 & 0.8333 & 0.7869  \\
    \midrule
    平均值 & 0.8759 & 0.8756 & 0.8444 & 0.8541 & 0.8127 & 0.8683 & 0.8908 & 0.8759 \\
    \midrule
    相比Baseline & -0.0366 & -0.0369 & -0.0681 & -0.0584 & \textbf{-0.0998} & -0.0442 & -0.0217 & -0.0366 \\
    \bottomrule
    \end{tabular}
    \label{tb:da_discuss_results}
\end{table}

\begin{figure}[h]
    \centering
    \subfloat[]{
        \includegraphics[width=0.45\linewidth]{loss_da.png}
        \label{train_process_da_discuss:loss}
    }
    \subfloat[]{
        \includegraphics[width=0.45\linewidth]{std_da.png}
        \label{train_process_da_discuss:std}
    }
    \caption{移除不同数据增强子策略的训练过程对比(a) 损失函数;(b) 提取特征的标准差}
    \label{train_process_da_discuss}
\end{figure}

\textbf{无标签数据集规模}

根据前文的讨论,SimSiam网络的性能可能在很大程度上依赖于编码器在无标签数据上进行自监督预训练的质量,现在讨论无标签样本规模对SimSiam网络性能的影响。在以下实验中,固定了训练集的样本规模为100,并分别设置了无标签数据集的规模为100、200、500、1000和2000。通过对比分析,评估了不同无标签数据集规模下模型的性能表现。从表\ref{tab:fine_tune_acc_ssv_size}和表\ref{tab:fine_tune_acc_ssv_size_pu}中可以看出,随着无标签数据集规模的增大,模型的性能普遍有所提升,尤其是在较大的数据集规模下,准确率的提升更为显著。

在无标签数据集规模为100时,SimSiam网络的准确率相对较低,尤其是在较大$\beta$值配置下(例如,$\beta=100$时,准确率仅为0.7200)。这一现象表明,在较小的数据集规模下,模型的自监督预训练效果不佳,未能有效地学习到足够的特征,从而导致性能较差。

随着无标签数据集规模的增大,模型的准确率逐渐提升(见图\ref{acc_vs_ssvsize}),尤其在数据集规模达到2000时,网络的表现达到最佳,准确率为0.9238。这表明,增加无标签数据集的规模对于模型性能的提升有显著作用。表格中的平均值进一步验证了这一点,随着数据集规模的增加,SimSiam网络的整体表现也逐步提高。在数据集规模为2000时,所有不平衡因子下的平均准确率均较高,凸显了大规模无标签数据集在自监督学习中的重要性。

然而,值得注意的是,当数据规模达到500及以上时,模型性能的提升趋于平稳。通过t-SNE分布图(图\ref{tsne_diff_ssv_size}),可以观察到特征提取质量的改善相对有限,特别是在数据规模从200增加到500时,性能提升最为显著,而从500到2000的提升则较为缓慢。这进一步表明,虽然大规模无标签数据集对模型性能有积极作用,但在一定规模后,性能提升的边际效益逐渐递减。

\begin{table}[h]
    \centering
    \caption{CWRU数据集不同无标签数据集规模配置下的SimSiam准确率/宏平均召回率}
    \begin{tabular}{cccccc}
    \toprule
    无标签数据集规模 & $\boldsymbol{\beta=1}$ & $\boldsymbol{\beta=10}$ & $\boldsymbol{\beta=50}$ & $\boldsymbol{\beta=100}$ & \textbf{平均值} \\
    \midrule
    100   & 0.8988 & 0.8942 & 0.8088 & 0.7200 & 0.8305 \\
    200   & 0.8710 & 0.8642 & 0.7479 & 0.7454 & 0.8071 \\
    500   & 0.9463 & 0.9577 & 0.8713 & 0.7929 & 0.8921 \\
    1000  & 0.9513 & 0.9508 & 0.9025 & 0.8746 & 0.9198 \\
    2000  & 0.9548 & 0.9538 & 0.8946 & 0.8919 & 0.9238 \\
    \bottomrule
    \end{tabular}
    \label{tab:fine_tune_acc_ssv_size}
\end{table}

\begin{table}[h]
    \centering
    \caption{PU数据集不同无标签数据集规模配置下的SimSiam准确率/宏平均召回率}
    \begin{tabular}{ccccccc}
    \toprule
    无标签数据集规模 & $\boldsymbol{\beta=1}$ & $\boldsymbol{\beta=10}$ & $\boldsymbol{\beta=50}$ & $\boldsymbol{\beta=100}$ & \textbf{平均值} \\
    \midrule
    100   & 0.6187  & 0.6221 & 0.5740 & 0.5213 & 0.5840 \\
    200   & 0.6297  & 0.6351 & 0.5815 & 0.5348 & 0.5953 \\
    500   & 0.6624  & 0.6704 & 0.6099 & 0.5549 & 0.6244 \\
    1000  & 0.6659  & 0.6656 & 0.6318 & 0.6103 & 0.6434 \\
    2000  & 0.6684  & 0.6677 & 0.6253 & 0.6243 & 0.6464 \\
    \bottomrule
    \end{tabular}
    \label{tab:fine_tune_acc_ssv_size_pu}
\end{table}

\begin{figure}[h]
    \centering
    \includegraphics[width=12cm]{acc_vs_ssvsize.png}
    \caption{不同无标签数据集规模配置下的SimSiam准确率/宏平均召回率变化图}
    \label{acc_vs_ssvsize}
\end{figure}


\begin{figure}[h]
    \centering
    \subfloat[]{
        \includegraphics[width=0.3\linewidth]{tsne_ssv_size_10.png}
    }
    \subfloat[]{
        \includegraphics[width=0.3\linewidth]{tsne_ssv_size_20.png}
    }
    \subfloat[]{
        \includegraphics[width=0.3\linewidth]{tsne_ssv_size_50.png}
    }
    \\
    \subfloat[]{
        \includegraphics[width=0.3\linewidth]{tsne_ssv_size_100.png}
    }
    \subfloat[]{
        \includegraphics[width=0.3\linewidth]{tsne_ssv_size_150.png}
    }
    \subfloat[]{
        \includegraphics[width=0.3\linewidth]{tsne_ssv_size_200.png}
    }
    
    \caption{不同无标签数据规模的SimSiam网络输出特征的T-SNE分布图:(a) 数据规模为100;(b) 数据规模为200;(c) 数据规模为500;(d) 数据规模为1000;(e) 数据规模为1500;(f) 数据规模为2000}
    \label{tsne_diff_ssv_size}
\end{figure}

\textbf{Batch Size}

文献\cite{chen2020simple}提到,较大的批大小(下称Batch Size)能够显著提高对比学习模型的性能。在本节中,通过实验探讨了不同Batch Size配置下SimSiam网络的表现。表\ref{tab:batch_size_simsiam}展示了不同Batch Size(16, 32, 64, 128, 256, 512)下,SimSiam网络在不同不平衡因子下的准确率。

对于$\beta=1$和$\beta=10$的情况,SimSiam网络在不同Batch Size下的表现相对稳定,唯一的显著性能下降发生在Batch Size为512且$\beta=1$时。这表明,较大的Batch Size可能对模型的训练稳定性产生负面影响,尤其是在低不平衡因子的配置下,可能导致训练过程中出现不稳定现象。

另一方面,针对$\beta=50$和$\beta=100$的配置,当Batch Size较小(16和32)时,模型的准确率表现出较大的波动。这种波动可能与小Batch Size导致的训练不充分或梯度估计不稳定有关。然而,当Batch Size增大至64及以上时,模型表现趋于稳定,说明较大的Batch Size有助于减小训练过程中波动,特别是在高不平衡因子的情况下。

从t-SNE分布图(图\ref{tsne_simsiam_diff_batch_size})中也可以观察到,当Batch Size≥64时,特征提取的质量保持稳定,各个类别的边界更加清晰;而在Batch Size为16或32时,部分点的分布呈现出交叠现象,表明个别点特征分离度较差。综合来看,SimSiam网络的性能对Batch Size的依赖相对较小,整体表现较为稳定,这进一步证明了在一定范围内,Batch Size的增大能够优化训练过程并稳定模型的性能。

反观 SimCLR 在不同 Batch Size 下的表现(见图 \ref{acc_of_batch_size}),随着 Batch Size 的增大,SimCLR模型的准确率呈现出明显的下降趋势,而SimSiam和BYOL的表现稳定,BYOL的性能在Batch Size为128时达到峰值,且是所有不平衡因子下都为最高性能(见表\ref{tab:batch_size_byol_finetune})。从表\ref{tab:batch_size_simclr_finetune}也可以看到,当Batch Size增大,SimCLR在不同不平衡因子的准确率全线下降。表明较大的 Batch Size 可能导致SimCLR模型学习到的特征质量下降。

此外,通过观察 t-SNE 分布图(见图 \ref{tsne_simclr_diff_batch_size}),进一步验证了这一现象。在较小的 Batch Size 下,特征的分布较为紧凑,且样本间的区分度较高。然而,随着 Batch Size 的增大,特征分布逐渐变得难以区分,特征提取的质量显著下降。据推测,这一变化可能与较大 Batch Size 与样本类别总数之比有关。在 SimCLR 中,同一批次中的样本通常被视为负样本,而当 Batch Size 增大时,同一批次中同类别样本的比例更高,从而导致模型倾向于增大同类样本之间的距离,这与对比学习的目标相悖——即相似样本应该聚集。由此猜测过大的 Batch Size 可能会损害对比学习的训练效果。

\begin{figure}[h]
    \centering
    \includegraphics[width=12cm]{acc_of_batch_size.png}
    \caption{不同Batch Size下SimSiam,SimCLR,BYOL的准确率/宏平均召回率变化图}
    \label{acc_of_batch_size}
\end{figure}

\begin{table}[h]
    \centering
    \caption{不同Batch Size配置下的SimSiam准确率/宏平均召回率}
    \begin{tabular}{cccccc}
    \hline
    \textbf{Batch Size} & \textbf{$\beta = 1$} & \textbf{$\beta = 10$} & \textbf{$\beta = 50$} & \textbf{$\beta = 100$} & \textbf{平均值} \\
    \hline
    16   & 0.9554 & 0.9567 & 0.9188 & 0.8798 & 0.9277 \\
    32   & 0.9492 & 0.9519 & 0.8600 & 0.7967 & 0.8895 \\
    64   & 0.9513 & 0.9508 & 0.9025 & 0.8746 & 0.9198 \\
    128  & 0.9433 & 0.9498 & 0.8804 & 0.8521 & 0.9064 \\
    256  & 0.9508 & 0.9692 & 0.8900 & 0.8131 & 0.9058 \\
    512  & 0.9258 & 0.9421 & 0.8748 & 0.8435 & 0.8966 \\
    \hline
    \end{tabular}
    \label{tab:batch_size_simsiam}
\end{table}

\begin{table}[h]
    \centering
    \caption{不同Batch Size配置下的SimCLR准确率/宏平均召回率}
    \begin{tabular}{cccccc}
    \hline
    \textbf{Batch Size} & \textbf{$\beta = 1$} & \textbf{$\beta = 10$} & \textbf{$\beta = 50$} & \textbf{$\beta = 100$} & \textbf{平均值} \\
    \hline
    16   & 0.9208 & 0.9423 & 0.9133 & 0.8760 & 0.9131 \\
    32   & 0.9108 & 0.9090 & 0.8556 & 0.8010 & 0.8691 \\
    64   & 0.8402 & 0.8535 & 0.8323 & 0.7860 & 0.8280 \\
    128  & 0.7935 & 0.8038 & 0.7388 & 0.7073 & 0.7609 \\
    256  & 0.7100 & 0.7292 & 0.6229 & 0.5617 & 0.6560 \\
    512  & 0.7185 & 0.7275 & 0.5940 & 0.5448 & 0.6462 \\
    \hline
    \end{tabular}
    \label{tab:batch_size_simclr_finetune}
\end{table}

\begin{table}[h]
    \centering
    \caption{不同Batch Size配置下的BYOL微调准确率/宏平均召回率}
    \begin{tabular}{cccccc}
    \hline
    \textbf{Batch Size} & \textbf{$\beta = 1$} & \textbf{$\beta = 10$} & \textbf{$\beta = 50$} & \textbf{$\beta = 100$} & \textbf{平均值} \\
    \hline
    16   & 0.8977 & 0.8977 & 0.8354 & 0.7881 & 0.8547 \\
    32   & 0.8904 & 0.8942 & 0.8481 & 0.8112 & 0.8610 \\
    64   & 0.9200 & 0.9050 & 0.8646 & 0.8029 & 0.8731 \\
    128  & 0.9392 & 0.9529 & 0.8846 & 0.8712 & 0.9120 \\
    256  & 0.9248 & 0.9246 & 0.8517 & 0.8644 & 0.8914 \\
    500  & 0.9267 & 0.9213 & 0.8179 & 0.7946 & 0.8651 \\
    \hline
    \end{tabular}
    \label{tab:batch_size_byol_finetune}
\end{table}
\begin{figure}[h]
    \centering
    \subfloat[]{
        \includegraphics[width=0.25\linewidth]{simsiam_batch_size_16.png}
    }
    \subfloat[]{
        \includegraphics[width=0.25\linewidth]{simsiam_batch_size_32.png}
    }
    \subfloat[]{
        \includegraphics[width=0.25\linewidth]{simsiam_batch_size_64.png}
    }
    \\
    \subfloat[]{
        \includegraphics[width=0.25\linewidth]{simsiam_batch_size_128.png}
    }
    \subfloat[]{
        \includegraphics[width=0.25\linewidth]{simsiam_batch_size_256.png}
    }
    \subfloat[]{
        \includegraphics[width=0.25\linewidth]{simsiam_batch_size_512.png}
    }
    
    \caption{不同Batch Size配置下的SimSiam网络输出特征的T-SNE分布图:(a) Batch Size为16;(b) Batch Size为32;(c) Batch Size为64;(d) Batch Size为128;(e) Batch Size为256;(f) Batch Size为512}
    \label{tsne_simsiam_diff_batch_size}
\end{figure}

\begin{figure}[h]
    \centering
    \subfloat[]{
        \includegraphics[width=0.25\linewidth]{simclr_batch_size_16.png}
    }
    \subfloat[]{
        \includegraphics[width=0.25\linewidth]{simclr_batch_size_32.png}
    }
    \subfloat[]{
        \includegraphics[width=0.25\linewidth]{simclr_batch_size_64.png}
    }
    \\
    \subfloat[]{
        \includegraphics[width=0.25\linewidth]{simclr_batch_size_128.png}
    }
    \subfloat[]{
        \includegraphics[width=0.25\linewidth]{simclr_batch_size_256.png}
    }
    \subfloat[]{
        \includegraphics[width=0.25\linewidth]{simclr_batch_size_512.png}
    }
    
    \caption{不同Batch Size配置下的SimCLR网络输出特征的T-SNE分布图:(a) Batch Size为16;(b) Batch Size为32;(c) Batch Size为64;(d) Batch Size为128;(e) Batch Size为256;(f) Batch Size为512}
    \label{tsne_simclr_diff_batch_size}
\end{figure}

\begin{figure}[h]
    \centering
    \subfloat[]{
        \includegraphics[width=0.25\linewidth]{byol_batch_size_16.png}
    }
    \subfloat[]{
        \includegraphics[width=0.25\linewidth]{byol_batch_size_32.png}
    }
    \subfloat[]{
        \includegraphics[width=0.25\linewidth]{byol_batch_size_64.png}
    }
    \\
    \subfloat[]{
        \includegraphics[width=0.25\linewidth]{byol_batch_size_128.png}
    }
    \subfloat[]{
        \includegraphics[width=0.25\linewidth]{byol_batch_size_256.png}
    }
    \subfloat[]{
        \includegraphics[width=0.25\linewidth]{byol_batch_size_512.png}
    }
    
    \caption{不同Batch Size配置下的BYOL网络输出特征的T-SNE分布图:(a) Batch Size为16;(b) Batch Size为32;(c) Batch Size为64;(d) Batch Size为128;(e) Batch Size为256;(f) Batch Size为512}
    \label{tsne_byol_diff_batch_size}
\end{figure}
\FloatBarrier  % 阻止后续浮动体越过这条线
\subsection{准确率、损失函数和特征标准差之间的相关性分析}

准确率、损失函数以及特征标准差是评估机器学习和深度学习的模型性能和学习过程的重要指标。准确率作为衡量模型预测正确性的标准,通常用于反映模型在分类任务中的表现;而损失函数则用来度量模型输出与真实标签之间的差距,指导优化器调整模型参数;特征标准差则能够从一定程度上反映模型“塌陷”的程度,同时反映特征提取的质量。

尽管这些指标分别在不同方面评价模型的表现,但它们之间是否存在内在的相关性,以及这种相关性是否能够提供更深刻的洞察,仍然是值得探讨的问题。准确率与损失函数之间的关系通常较为直观:随着损失函数的减少,模型的预测精度有望提高。然而,特征标准差在此过程中所起的作用可能更加复杂。特征标准差的变化可能在一定程度上影响模型对数据分布的适应能力,进而间接影响最终的准确率。

因此,本节旨在深入探讨准确率、损失函数和特征标准差之间的相关性,通过量化这些指标的关系,揭示它们在模型训练中的互动机制,以便为模型优化和性能提升提供理论依据。通过这一分析,可以更全面地理解训练过程中各项指标之间的相互作用,进而指导实际应用中的模型调整与改进。

\textbf{Loss与Std:}图\ref{correlation_with_loss_std}显示了图\ref{train_process_da_discuss}中的Loss曲线与标准差曲线的皮尔森相关系数(Pearson Correlation)与斯皮尔曼相关系数(Spearman Correlation)。以下是对结果的分析:

在Baseline中,模型包含了所有的数据增强策略,Pearson相关系数为-0.9448,Spearman相关系数为-0.9232。这表明,在包含所有增强策略的情况下,Loss与标准差之间有较强的负相关性。这个相关性表明,随着训练过程中的损失降低,特征的标准差也相应增大,表明模型的提取特征的能力越来越稳定。

Pearson相关系数和Spearman相关系数在许多情况下呈现出相似的趋势,这表明数据增强策略对Loss和标准差的关系影响较为一致。从结果可以看出,不同的增强策略对Loss与标准差的相关性产生了不同的影响。总的来说,删除高斯噪声、相位扰动和缩放等增强策略在影响Loss与标准差的关系时,表现出了较强的负相关性,表明这些操作增加了数据的多样性或扰动,使得模型更容易适应各种尺度和噪声变化。而像竖直翻转和水平翻转等操作对Loss与标准差的影响较小,表明这些操作对于周期性信号的处理可能并没有显著改变数据的基本特征。因此,在设计数据增强策略时,应该根据具体任务的特性选择合适的增强方法,以最大程度提升模型的鲁棒性和泛化能力。

同时,图\ref{correlation_with_loss_std}显示,SimSiam在去除Stop-Grad模块或Predictor模块后,表现出较高的正相关性,尤其是斯皮尔曼相关系数几乎接近1,这表明准确率与损失/标准差之间呈现较强的等级相关性。这意味着去除这些模块后,损失和标准差的变化对模型性能的影响呈现出更加明显的等级趋势。具体来说,去除Stop-Grad模块后,模型的性能与损失、标准差之间的关系变得更加显著,尤其在SimSiam w/o SG中,斯皮尔曼相关系数高达0.9935,接近1,说明这种关联非常强。

对于去除Predictor模块的情况,尽管皮尔森相关系数仍然较高(0.9735),但斯皮尔曼相关系数较低(0.7907),这表明去除Predictor模块后,损失和标准差的变化对准确率的影响在等级上的一致性有所减弱。这个变化可能揭示了模型在去除Predictor模块后表现出的一定程度的性能下降,尤其是在捕捉复杂模式时的能力受限。

这个相关性也揭示了模型在去除Stop-Grad模块或Predictor模块后的潜在模型崩塌现象。去除这些模块后,模型的性能变化更加显著,且损失与准确率的关系变得更加直接,暗示着模型在这些模块缺失的情况下,可能缺乏必要的学习能力和稳定性,导致性能严重下降或过拟合。这种分析进一步表明,Stop-Grad模块和Predictor模块在SimSiam模型中的重要性,去除这些模块后可能影响模型的稳定性和泛化能力,进而导致性能的大幅波动和模型崩塌现象。这与前文的实验结论相一致。


% \begin{table}
%     \centering
%     \caption{不同增强子策略的Loss与Standard Deviation的相关系数}
%     \begin{tabular}{ccc}
%         \hline
%         \textbf{删除的增强子策略/模块变化} & \textbf{Pearson Correlation} & \textbf{Spearman Correlation} \\
%         \hline
%         Baseline & -0.9448 & -0.9232 \\
%         高斯噪声 (DA0)  & -0.9619 & -0.8994 \\
%         相位扰动 (DA1)  & -0.9169 & -0.9365 \\
%         块打乱 (DA2)  & -0.9288 & -0.8851 \\
%         掩码 (DA3)  & -0.8962 & -0.8904 \\
%         缩放 (DA4)  & -0.9775 & -0.8636 \\
%         绝对值 (DA5)  & -0.9412 & -0.9064 \\
%         竖直翻转 (DA6)  & -0.9500 & -0.9060 \\
%         水平翻转 (DA7)  & -0.8457 & -0.6589 \\
%         SimSiam w/o SG & 0.8853 & 0.9935 \\
%         SimSiam w/o Pred & 0.9735 & 0.7907 \\
%         \hline
%     \end{tabular}
%     \label{tb:correlation_with_loss_std}
% \end{table}
\begin{figure}[h]
    \centering
    \includegraphics[width=1.0\linewidth]{da_correlation.png}
    \caption{不同增强子策略的Loss与Standard Deviation的相关系数}
    \label{correlation_with_loss_std}
\end{figure}
\textbf{准确率与Loss/Std:}表\ref{tab:correlation_with_acc}展示了准确率与损失函数以及标准差之间的相关性分析。具体分析如下:

首先,准确率与损失函数的相关性分析表明,无论是使用皮尔森相关系数(-0.0048)还是斯皮尔曼相关系数(-0.1557),两者的相关系数都非常接近零,且对应的 p-value 分别为 0.9910 和 0.7128。这些结果表明,准确率与损失函数之间几乎没有线性或等级上的相关性。因此,尽管损失值反映了模型在训练过程中的误差大小,但它与最终的准确率之间并没有明显的联系。

其次,准确率与标准差之间的相关性也表现出较弱的关系。皮尔森相关系数为 0.1972,斯皮尔曼相关系数为 0.2994,二者均较小,说明准确率与标准差之间的线性或等级相关性较弱。同时,p-value 分别为 0.6398 和 0.4713,这意味着这些相关性并不显著。因此,标准差可能未能有效反映模型预测的波动性与准确率之间的关系。

综合来看,表格中的结果表明,准确率与损失以及标准差之间的相关性较低,无论是线性还是非线性关系均不显著。这可能意味着,尽管损失和标准差是训练过程中非常重要的指标,但它们并不能直接有效地预测最终模型的准确率。在实际应用中,可能需要依赖其他更直接的性能指标来评估模型的表现。

\begin{table}[h]
    \centering
    \caption{SimSiam准确率与不同指标的相关性分析}
    \begin{tabular}{cccc}
    \toprule
    \textbf{指标} & \textbf{相关系数类型} & \textbf{相关系数} & \textbf{p-value} \\
    \midrule
    准确率与Loss     & 皮尔森相关系数   & -0.0048 & 0.9910 \\
    准确率与Loss     & 斯皮尔曼相关系数 & -0.1557 & 0.7128 \\
    准确率与标准差      & 皮尔森相关系数   & 0.1972 & 0.6398 \\
    准确率与标准差      & 斯皮尔曼相关系数 & 0.2994 & 0.4713 \\
    \bottomrule
    \end{tabular}
    \label{tab:correlation_with_acc}
    \end{table}
\FloatBarrier  % 阻止后续浮动体越过这条线
\section{本章小结}
本章节提出了一种结合简单暹罗孪生网络(SimSiam)与基于 CMA-ES 优化器搜索最优数据增强策略的对比学习自监督预训练模型架构,并在构建的符合帕累托长尾分布的 CWRU 数据集和PU数据集上取得了显著的性能提升。通过实验,系统地分析了模型架构中各模块的作用,并从理论上揭示了 Predictor 模块与 Stop-Grad 模块之间的内在联系。与当前SOTA的方法对比,进一步验证了所提出模型的有效性。此外,通过消融实验,系统探讨了无标签数据集的引入、Batch Size 等关键参数对模型性能的影响。
然而,研究过程中发现,模型在微调阶段仍然受到由样本分布不平衡引起的偏差影响。因此,下一章将重点针对这一问题提出改进方案,以进一步提升模型的分类性能。

\chapter{半监督学习与边距调整改进的暹罗孪生故障诊断网络}
上一章展示了自监督学习范式下的二阶段式训练方法(如图\ref{self_supervise_procedure}所示),其在故障诊断任务的长尾学习场景中显著提升了模型性能。然而,在微调阶段,样本分布的不平衡问题仍对模型的性能产生不利影响,导致分类器的决策边界发生偏移。如图\ref{decision_edge_bias}所示,决策边界向尾部类特征空间发生偏移,使头部类在特征空间中占据更大的体积,从而降低了尾部类的分类准确率。本章节将重点探讨一种结合半监督学习与MARC(Margin Calibration)决策边界调整的微调算法,以缓解长尾分布对微调阶段决策边界偏移的影响,并进一步提升尾部类的识别能力。在处理长尾分布和样本不平衡问题时,决策面调整方法(MARC)提供了一种有效的方式来改善分类器的性能,特别是在少数类别(尾部类别)上。文献\cite{wang2023margin}介绍的决策面调整算法通过引入边距校准方法,调整分类器的决策面,使其更加均衡地处理各个类别的数据。研究发现长尾学习中的边距(Margin)和 logits 存在偏差,通过训练$\omega$ 和 $\beta$两个参数校准这种偏差。然而,该方法未能充分利用无标签数据的信息,限制了其在实际应用中的性能发挥。因此,本文结合半监督学习的思想,对该算法进行改进,以进一步提升模型的泛化能力。
\begin{figure}[h]
    \centering
    \includegraphics[width=12cm]{decision_edge_bias.pdf}
    \caption{由长尾分布带来的决策面偏移示意图}
    \label{decision_edge_bias}
\end{figure}
\FloatBarrier  % 阻止后续浮动体越过这条线

\section{模型架构}
\FloatBarrier  % 阻止后续浮动体越过这条线
\subsection{半监督学习与边距调整改进的暹罗孪生故障诊断网络整体架构}

针对原有MARC算法不能充分运用无监督样本的信息,本研究在MARC算法中引入半监督学习的思想,优化了上一章的SimSiam架构的微调过程,提出了一种结合两者的 SimSiam 模型架构,称为 基于半监督MARC 微调的 SimSiam 模型(Semi-MARC SimSiam)。该模型通过三阶段的逐步训练和优化,旨在解决长尾分布和样本不平衡问题,并显著提升分类器在尾部类别上的性能。其具体模型训练架构流程如图 \ref{semi_marc_simsiam} 所示。
该模型架构包括以下四个阶段:

\begin{enumerate}[label={(\arabic*)}]
    \item \textbf{阶段一(自监督学习)}:  
    在第一阶段,基于 SimSiam 架构对模型进行预训练。具体而言,利用大量无标签样本和少量有标签的长尾样本,训练一个性能较优的特征提取网络和初步分类器。SimSiam 的自监督特性能够在数据量受限的情况下充分挖掘数据的潜在分布信息,为后续的半监督学习提供高质量的特征表示。

    \item \textbf{阶段二(半监督学习)}:  
    在第二阶段,基于阶段一训练得到的 SimSiam 模型,使用其生成无标签样本的伪标签(Pseudo Labels),将伪标签视为无标签数据的“真实标签”。随后,将补充有伪标签的无标签数据与原始有标签数据集相结合,通过最小化标记数据和伪标签数据的损失函数,重新优化分类器的全连接层参数,得到半监督微调后的Semi-SimSiam模型。这一阶段的目标是榨取无标签数据集的信息,扩充原有数据集。

    \item \textbf{阶段三(半监督优化的MARC 决策边界调整)}:  
    在第三阶段,利用经过半监督学习优化后的分类器及其输出的 logits和经伪标签扩充的数据集,对决策边界进行微调。具体而言,通过训练两个向量参数 $\omega$ 和 $\beta$,分别控制分类器输出的尺度因子和偏置项,调整分类器的决策面,以提升模型在尾部类别上的分类准确率。MARC 的引入有效平衡了模型在头部类别和尾部类别上的性能差异。

    \item \textbf{阶段四(故障诊断)}:  
    最终,将阶段二训练得到的 Semi-SimSiam 模型与阶段三中学习到的尺度因子 $\omega$ 和偏置项 $\beta$ 相结合,构建完整的故障诊断模型Semi-Marc SimSiam。该模型综合利用自监督特征学习、伪标签生成与决策边界优化三者的优势,对输入样本进行精确分类,从而实现更为稳健且性能均衡的故障诊断。
\end{enumerate}

\begin{figure}[h]
    \centering
    \includegraphics[width=14cm]{semi_marc_simsiam.pdf}
    \caption{基于半监督学习和 MARC 微调的 SimSiam 模型流程框图}
    \label{semi_marc_simsiam}
\end{figure}

整体流程通过结合 SimSiam 的特征提取能力、半监督学习对伪标签的利用,以及 MARC 算法对决策边界的微调,形成了一个多阶段优化框架。该方法在处理长尾分布和样本不平衡问题时,展现出显著的优势,尤其是在提高尾部类别分类准确率方面效果显著。
\FloatBarrier  % 阻止后续浮动体越过这条线
\subsection{半监督学习模型架构}
本研究提出了一种结合半监督学习与决策边界调整算法的框架,以有效解决长尾分布和样本不平衡问题。受到文献\cite{yang2020rethinking,wang2023margin}的启发,研究借鉴了半监督学习(Semi-Supervised Learning,下称Semi)的经典流程(如图\ref{semi_supervise_procedure}所示)。该流程分为两个阶段:

在第一阶段,使用原始的不平衡标记数据集 $\mathcal{D}_{L}$ 训练一个初步分类器 $f_{\hat{\theta}}$。随后,在第二阶段,利用该分类器对未标记数据集 $\mathcal{D}_{U}$ 进行推断并生成伪标签 $\hat{y}$。伪标签数据与标记数据结合,通过最小化以下损失函数优化模型:
\begin{equation} 
    \mathcal{L}(\mathcal{D}_{L},\theta) + \omega \mathcal{L}(\mathcal{D}_{U},\theta), 
    \label{eq:semi_loss}
\end{equation}
其中,$\omega$ 为未标记数据的损失权重,$\mathcal{L}(\mathcal{D}_{L},\theta)$ 和 $\mathcal{L}(\mathcal{D}_{U},\theta)$ 分别代表基于标记数据和伪标签数据的损失项。通过整合这两部分损失,最终得到优化后的分类器 $f_{\hat{\theta_f}}$,以更好地建模数据分布,尤其是尾部类别(少数类别)。该过程通过充分利用未标记数据 $\mathcal{D}_{U}$,提升模型对少数类别的分类能力,从而显著优化其决策边界。

\begin{figure}[h]
    \centering
    \includegraphics[width=12cm]{semi_supervise_procedure.pdf}
    \caption{半监督学习流程图}
    \label{semi_supervise_procedure}
\end{figure}
\FloatBarrier  % 阻止后续浮动体越过这条线
\subsection{经半监督优化的决策面调整算法架构}
与此同时,在处理长尾分布和样本不平衡问题时,决策面调整方法(MARC)提供了一种有效的方式来改善分类器的性能,特别是在少数类别(尾部类别)上。文献\cite{wang2023margin}介绍的决策面调整算法通过引入边距校准方法,调整分类器的决策面,使其更加均衡地处理各个类别的数据。

研究发现长尾学习中的边距(Margin)和 logits 存在偏差。边距在图\ref{marc_illustration}中进行了说明。定义类别 $ j $ 的仿射超平面 $ H_j \in \mathbb{R}^{p-1} $ 为 $ W_j z + b_j = 0 $,即任何落在超平面 $ H_j $ 正侧的表示点可以归类为类别 $ j $。假设 $ z_0 $ 是一个满足 $ W_j z_0 + b_j = 0 $ 的点,即 $ z_0 $ 在超平面 $ H_j $ 上。假设 $ z_1 $ 是特征空间中的任意点。构造从 $ z_0 $ 指向 $ z_1 $ 的向量 $ z_1 - z_0 $,并将其投影到法向量 $ W_j $ 上。投影向量 $ \text{proj}_{W_j}(z_1 - z_0) $ 的长度即是从 $ z_1 $ 到 $ H_j $ 的边距。边距$d_j$的计算如下:

\begin{equation}
    \begin{split}
        d_j &= \left\| \text{proj}_{W_j}(z_1 - z_0) \right\| \\
            &= \left\| \frac{W_j \cdot (z_1 - z_0)}{\|W_j\|} \right\| \\
            &= \frac{W_j \cdot z_1 - W_j \cdot z_0}{\|W_j\|} \\
            &= \frac{W_j z_1 + b_j}{\|W_j\|} \quad (\text{由于 } W_j z_0 + b_j = 0)
    \end{split}
\end{equation}
其中 $ \| \cdot \| $ 表示 L2 范数。因此,logit $= W_j \cdot z_1 + b_j $ 也可以表示为 $ \|W_j\| d_j $。
\begin{figure}[h]
    \centering
    \includegraphics[width=12cm]{marc_illustration.pdf}
    \caption{决策面与决策边界示意图,红色点代表头部类,蓝色点代表尾部类}
    \label{marc_illustration}
\end{figure}

MARC算法包括两个关键参数:$\omega$ 和 $\beta$,其中 $\omega$ 是用于调整输出的尺度因子,而 $\beta$ 则是用于调整输出的偏置项。调整后的边距$\hat{d}_j$如下:
\begin{equation}
    \hat{d}_j = \omega_j \cdot d_j + \beta_j
\end{equation}
这两个参数通过最小化类平衡的损失函数(如CB LOSS)来进行优化,从而调整获得适合当前任务的最优决策面。校准后的logits通过下式计算
\begin{equation}
    \begin{split}
    logits_{marc} = \| \mathbf{W}_j \| \hat{d}_j &= \| \mathbf{W}_j \| (\omega_j \cdot d_j + \beta_j) \\
    &= \omega_j \cdot \| \mathbf{W}_j \| d_j + \beta_j \cdot \| \mathbf{W}_j \| \\
    &= \omega_j \cdot \eta_j + \beta_j \cdot \| \mathbf{W}_j \|,
    \end{split}
\end{equation}
其中$\eta_j$是原始的logits。算法的具体实现步骤参考表\ref{alg:marc}。在MARC伪代码中,$\omega$ 和 $\beta$ 是通过torch.nn.Parameter初始化为可训练的参数,它们分别表示边距的缩放因子和偏置项。使用torch.no\_grad()关键字来避免计算梯度,确保在前向传播时不会影响到梯度计算。接下来,代码通过计算网络的权重范数以及模型在给定输入 $x$ 上的输出结果,来调整输出的决策面,生成校准后的输出 $\text{logit\_after}$。在实际应用中,MARC 算法有助于减小尾部类别的分类误差,尤其是在样本不平衡的情况下,通过有效地调整决策边距来改善模型的泛化能力和分类精度。
\begin{table}[h]
    \centering
    \begin{tabular}{@{}l@{}} % 使用 @{} 去掉默认的左右边距,l 表示左对齐
    \toprule
    \multicolumn{1}{@{}l@{}}{\textbf{MARC伪代码(用Pytorch描述)}} \\ % 左对齐文本
    \midrule
    \begin{lstlisting}[basicstyle=\ttfamily,frame=none]
1: 初始化边界校准方法:
    omega = torch.nn.Parameter(torch.ones(1, num_classes))
    beta = torch.nn.Parameter(torch.zeros(1, num_classes))

2: 输入:训练数据 x(由标签数据集D_L和带伪标签的无标签数据集D_U组成),
   标准的预训练神经网络模型。

3: with torch.no_grad():
4:     w_norm = torch.norm(model.fc.weight, dim=1)
5:     logit_before = model(x)
6:     logit_after = omega * logit_before + beta * w_norm
7: 计算损失并更新参数 omega 和 beta。
    \end{lstlisting} \\
    \hline
    \end{tabular}
    \caption{MARC 算法的实现伪代码}
    \label{alg:marc}
\end{table}
\FloatBarrier  % 阻止后续浮动体越过这条线
\subsection{实验设置}
\textbf{模型配置}:阶段一SimSiam的网络设置与上一章相同。在决策面调整过程中,将 $\omega$ 初始化为大小为 $1 \times \text{num\_classes}$ 的全 $1$ 向量,而 $\beta$ 则初始化为相同大小的全 $0$ 向量。

\textbf{优化器}:本研究使用 SGD 优化器训练模型。网络参数通过公式 (\ref{eq:SGD}) 更新。学习率初始设置为 \( 0.05 \times \frac{\text{BatchSize}}{256} \),学习率采用余弦衰减调度。余弦衰减调度的公式如式(\ref{eq:cos_decay})所示。

\textbf{损失函数}:阶段一SimSiam对比学习使用的损失函数如上一章所介绍。阶段二、阶段三的损失函数为式(\ref{eq:semi_loss}),其中$\mathcal{L}$为交叉熵损失函数式(\ref{eq:cross_entropy}),$\omega$设为0.9。

\textbf{数据细节}:
无标签数据集的规模为类别数*100,有标签数据集为不同不平衡因子构造帕累托分布的数据,测试集为均匀分布。每个样本包含 1024 个数据点。每个阶段分别训练150个周期。
在整个训练过程中,每个输入模型的 mini-batch 大小设定为 64。
\FloatBarrier  % 阻止后续浮动体越过这条线
\section{实验与分析}

\subsection{在 CWRU 数据集和 PU 数据集上的实验结果}
本研究基于CWRU数据集和PU数据集进行实验,数据集的构造方式与\ref{simsiam_cwru_results}节描述一致。每个训练阶段均运行150个周期。不同不平衡因子条件下的分类准确率结果如表\ref{simsiam_semi_results}所示。图\ref{fig:semi_marc_class_acc_for_beta}展示了在不同$\beta$值下,采用Semi-Marc微调方法后,各类别准确率在三个训练阶段(SimSiam、Semi、Marc)中的变化趋势。值得注意的是,Class 0为头部类别,Class 9为尾部类别,各类别样本数量呈递减趋势。

表 \ref{simsiam_semi_results} 与表 \ref{simsiam_semi_results_pu} 分别展示了不同模型在 CWRU 与 PU 数据集上,针对不同不平衡因子 $\beta$ 所获得的准确率与宏平均召回率。从整体结果来看,本文提出的 Semi-Marc SimSiam 模型在两个数据集上均取得了最优的平均表现,分别达到 0.9454 和 0.6614,显著优于基础模型 SimSiam 及其两种单独增强版本。这表明,结合半监督机制与 MARC 特征增强策略能够有效缓解样本不平衡问题所带来的性能退化,提升模型的鲁棒性与泛化能力。

在 CWRU 数据集中,随着不平衡因子逐渐增大,所有模型的性能均呈现一定程度下降,其中基础的 SimSiam 模型在 $\beta=100$ 时准确率降至 0.8450,而 Semi-Marc SimSiam 在相同条件下依然维持在 0.9142,表现出对极端不平衡场景更强的适应能力。与之相对,PU 数据集整体难度更大,所有模型的准确率普遍低于 CWRU,反映出该数据集的复杂性和挑战性。然而,即使在此背景下,Semi-Marc SimSiam 依然在所有 $\beta$ 设置中保持领先,且波动幅度较小,说明其具有更好的稳定性与迁移能力。

进一步对比可以发现,单独引入半监督策略或 MARC 技术虽能在一定程度上改善模型性能,但其提升幅度相较联合方法较为有限,尤其是在高度不平衡情形下难以维持较高的准确率。这说明两者在机制上具有互补性:半监督策略能够引导模型更充分地利用未标记样本以增强泛化能力,而 MARC 则能通过特征层的增强提升模型对少数类的区分能力,从而在联合使用时实现协同增益,显著提升分类性能。

从图\ref{simsiam_semi_results}可以清晰地观察到,尾部类别Class 8在经过Semi-Marc微调后,其分类准确率显著提升。此外,除了Class 8和Class 5外,其余类别的分类准确率不仅未下降,反而有所提升。具体变化趋势见图\ref{fig:semi_marc_class_acc_for_beta10}、图\ref{fig:semi_marc_class_acc_for_beta50}和图\ref{fig:semi_marc_class_acc_for_beta100},均验证了Semi-Marc微调方法在提升尾部类分类性能上的有效性,同时对头部类和中部类的分类性能也具有积极影响。

\begin{table}[h]
    \caption{CWRU数据集SimSiam 与 Semi-Marc SimSiam 在不同 $\beta$ 值下的准确率/宏平均召回率}
    \centering
    \begin{tabular}{cccccc}
    \toprule
    不平衡因子 $\beta$  & SimSiam & Semi SimSiam & Marc SimSiam & Semi-Marc SimSiam \\
    \midrule
    1   & 0.9515 & 0.9685 & 0.94896 & 0.9781 \\
    10  & 0.9402 & 0.9606 & 0.9575  & 0.9623 \\
    50  & 0.9131 & 0.9248 & 0.91687 & 0.9271 \\
    100 & 0.8450 & 0.9090 & 0.87896 & 0.9142 \\
    \midrule
    \textbf{平均值} & 0.9125 & 0.9407 & 0.9256 & \textbf{0.9454} \\
    \bottomrule
    \end{tabular}
    \label{simsiam_semi_results}
\end{table}

\begin{table}[h]
    \caption{PU数据集SimSiam 与 Semi-Marc SimSiam 在不同 $\beta$ 值下的准确率/宏平均召回率}
    \centering
    \begin{tabular}{cccccc}
    \toprule
    不平衡因子 $\beta$  & SimSiam & Semi SimSiam & Marc SimSiam & Semi-Marc SimSiam \\
    \midrule
    1   & 0.6661 & 0.6801 & 0.6635 & 0.6821 \\
    10  & 0.6581 & 0.6764 & 0.6696 & 0.6747 \\
    50  & 0.6392 & 0.6474 & 0.6408 & 0.6499 \\
    100 & 0.5915 & 0.6363 & 0.6152 & 0.6390 \\
    \midrule
    \textbf{平均值} & 0.6387 & 0.6601 & 0.6473 & \textbf{0.6614} \\
    \bottomrule
    \end{tabular}
    \label{simsiam_semi_results_pu}
\end{table}


\begin{figure}[h]
    \centering
    \subfloat[]{
        \includegraphics[width=0.45\linewidth]{class_acc_for_beta=1_10avg.png}
        \label{fig:semi_marc_class_acc_for_beta1}
    }
    \subfloat[]{
        \includegraphics[width=0.45\linewidth]{class_acc_for_beta=10.png}
        \label{fig:semi_marc_class_acc_for_beta10}
    }
    \par\medskip
    \subfloat[]{
        \includegraphics[width=0.45\linewidth]{class_acc_for_beta=50.png}
        \label{fig:semi_marc_class_acc_for_beta50}
    }
    \subfloat[]{
        \includegraphics[width=0.45\linewidth]{class_acc_for_beta=100.png}
        \label{fig:semi_marc_class_acc_for_beta100}
    }
    \caption{不同$\beta$值下基于Semi-Marc微调的各类别准确率/宏平均召回率的变化 (a) $\beta=1$;(b) $\beta=10$;(c) $\beta=50$;(d) $\beta=100$}
    \label{fig:semi_marc_class_acc_for_beta}
\end{figure}

Semi-MARC是一种架构,本身就可以作用于其他任何有无标签样本的网络。因此,以下研究了Semi-MARC对SimCLR,BYOL,CNN网络的作用。其中,SimCLR和BYOL取性能最高的Batch Size训练,即SimCLR的Batch Size为16,BYOL的Batch Size为128。

从图\ref{all_models_before_after_semimarc}结果可以观察到,四种模型在引入 Semi-Marc 方法之后均表现出不同程度的性能提升,尤其是 SimSiam 和 BYOL 的准确率显著提高,分别从 0.9131 提升至 0.9454,以及从 0.8610 提升至 0.9297。值得注意的是,在未引入 Semi-Marc 时,SimCLR 以 0.9265 的准确率略高于 SimSiam,但在引入 Semi-Marc 后,SimSiam 凭借更显著的性能增益反超 SimCLR,最终以 0.9454 的准确率成为表现最优的模型。这一现象表明 Semi-Marc 对于基于对比学习的结构具有良好的兼容性与提升效果,尤其在 SimSiam 这类不依赖负样本对比的轻量化结构中,其辅助作用更加明显。同时,传统的 CNN 模型也获得了一定的性能改善,从 0.7420 提升至 0.7783,说明 Semi-Marc 并非仅对自监督模型有效,也具备一定的通用性。整体而言,Semi-Marc 能够有效缓解数据不平衡对模型训练的影响,并提升模型在长尾分布数据集条件下的特征表达能力,其在多个结构中的广泛适配性进一步验证了其作为一类通用增强策略的潜力与价值。

总的来说,Semi-Marc作为一种有效的模型微调架构,不仅能够提升网络对抗长尾效应的鲁棒性,而且能够对不同网络架构产生明显的性能优化效果,这为自监督学习模型提供了一种有力的优化手段。

\begin{figure}[h]
    \centering
    \includegraphics[width=12cm]{all_models_before_after_semimarc.png}
    \caption{SimSiam,SimCLR,BYOL,CNN在Semi-Marc微调前后的平均准确率/宏平均召回率柱状图}
    \label{all_models_before_after_semimarc}
\end{figure}
\FloatBarrier  % 阻止后续浮动体越过这条线
\subsection{消融实验}
在本小节,基于CWRU数据集对训练过程中的参数进行了消融实验,主要包含各个阶段的“需要梯度”(Require Grad) 参数、无标签数据集规模、Batch Size、阶段一的网络模型。

\textbf{“需要梯度”}

编码器参数“需要梯度”(Require Grad)是一个布尔类型的参数,它代表了在训练中是否对该模块的网络权重进行反向传播更新。在SimSiam对比学习阶段和半监督学习阶段都需要对该参数进行设置,表\ref{tab:require_grad_discuss}对该参数进行了实验。
表中,fine\_tune\_requires\_grad 表示在 SimSiam 阶段是否对编码器(Encoder)的参数进行微调,semi\_requires\_grad 则表示在半监督学习阶段是否对编码器的参数进行微调。通过对比不同配置下的准确率,可以得出以下结论:

\begin{itemize}
    \item \textbf{SimSiam 阶段梯度更新对性能的影响}:当 fine\_tune\_requires\_grad 为 True 时,模型在 SimSiam 阶段对编码器参数进行微调。从表中可以看出,无论 semi\_requires\_grad 的配置如何,fine\_tune\_requires\_grad=True 时的平均准确率(0.9396 和 0.9398)略低于 fine\_tune\_requires\_grad=False 时的平均准确率(0.9409 和 0.9454)。这表明在 SimSiam 阶段冻结编码器参数可能有助于提升模型的整体性能。

    \item \textbf{半监督学习阶段梯度更新对性能的影响}:当 semi\_requires\_grad 为 False 时,模型在半监督学习阶段冻结编码器参数。从表中可以看出,semi\_requires\_grad为False 时的平均准确率(0.9398 和 0.9454)略高于 semi\_requires\_grad为True 时的平均准确率(0.9396 和 0.9409)。这表明在半监督学习阶段冻结编码器参数可能有助于进一步提升模型的性能。

    \item \textbf{综合配置的影响}:当 fine\_tune\_requires\_grad=False 且 semi\_requires\_grad=False 时,模型在两个阶段均冻结编码器参数,此时的平均准确率最高(0.9454)。这表明在 SimSiam 和半监督学习阶段均冻结编码器参数可能是最优的配置方案。
\end{itemize}

综上所述,通过对不同梯度配置的实验分析,可以得出在 SimSiam 和半监督学习阶段冻结编码器参数有助于提升模型的整体性能。这一发现为后续研究提供了重要的参考依据。

\begin{table}[h]
    \centering
    \caption{不同配置下 Semi-Marc 方法的准确率/宏平均召回率,fine\_tune\_requires\_grad表示是否在SimSiam微调时微调编码器,semi\_requires\_grad表示是否在Semi时微调编码器}
    \begin{tabular}{ccccccc}
    \toprule
    fine\_tune\_requires\_grad & semi\_requires\_grad & $\beta=1$ & $\beta=10$ & $\beta=50$ & $\beta=100$ & 平均值 \\
    \midrule
    True  & True  & 0.9727 & 0.9650 & 0.9185 & 0.9021 & 0.9396 \\
    True  & False & 0.9681 & 0.9669 & 0.9229 & 0.9013 & 0.9398 \\
    False & True  & 0.9746 & 0.9744 & 0.9175 & 0.8973 & 0.9409 \\
    False & False & 0.9781 & 0.9623 & 0.9271 & 0.9142 & \textbf{0.9454} \\
    \bottomrule
    \end{tabular}
    \label{tab:require_grad_discuss}
\end{table}

\textbf{无标签数据集规模}

由于无标签数据集的规模直接决定了半监督学习所扩增的带有伪标签的数据集大小,因此无标签数据集规模对Semi-Marc SimSiam模型性能的影响至关重要。在以下实验中,固定了训练集的样本规模为100,并分别设置了无标签数据集的规模为100、200、500、1000和2000。根据图\ref{acc_vs_ssvsize_semimarc}的实验结果,随着无标签数据集规模的增加,Semi-Marc SimSiam的性能逐步提高。这表明,较大的无标签数据集有助于提升模型在自监督预训练阶段学习到的特征质量。

然而,值得注意的是,当数据规模达到500及以上时,模型性能的提升趋于平稳,甚至在数据规模达到2000时,性能出现了微弱的下降。这一现象可能与以下几个因素有关:

\begin{enumerate}[label={(\arabic*)}]
    \item \textbf{数据冗余性:} 随着数据集规模的增大,训练数据的冗余性也随之增加,尤其是当无标签数据集的质量相对较低时,更多的无标签数据并没有提供足够的信息来进一步提升模型的性能。此时,模型可能受到冗余信息的干扰,导致训练效率降低,从而影响了性能提升。
    \item \textbf{数据质量:} 在某些情况下,较大的无标签数据集可能包含更多噪声数据,这些噪声数据可能对模型学习到的深层特征产生负面影响。因此,随着数据规模的增加,模型的性能可能不再持续提高,甚至出现性能下降的情况。
    \item \textbf{过拟合风险:} 当数据集规模过大时,模型可能过度依赖于训练数据中的某些特征,导致过拟合的风险增加,从而影响了模型在测试集上的泛化能力。
\end{enumerate}

综合来看,虽然增加无标签数据集的规模对模型性能有积极的促进作用,但当数据规模超过一定阈值后,性能提升趋于平稳,甚至可能出现下降。因此,在实际应用中,选择合适的无标签数据集规模,确保数据的多样性和质量,可能比单纯地增加数据量更为重要。

% \begin{table}[h]
%     \centering
%     \caption{不同无标签数据集规模和不平衡因子$\beta$配置下的Semi-Marc SimSiam准确率}
%     \begin{tabular}{ccccccc}
%     \hline
%     \textbf{无标签数据集规模} & \textbf{$\beta = 1$} & \textbf{$\beta = 10$} & \textbf{$\beta = 50$} & \textbf{$\beta = 100$} & \textbf{平均值} \\
%     \hline
%     100   & 0.8967 & 0.8975 & 0.8412 & 0.7840 & 0.8549 \\
%     200   & 0.8813 & 0.8706 & 0.7669 & 0.7777 & 0.8241 \\
%     500   & 0.9621 & 0.9698 & 0.8958 & 0.8444 & 0.9180 \\
%     1000  & 0.9602 & 0.9544 & 0.9146 & 0.8946 & 0.9310 \\
%     2000  & 0.9583 & 0.9538 & 0.9029 & 0.8967 & 0.9279 \\
%     \hline
%     \end{tabular}
%     \label{tab:semi_marc_ssv_size}
% \end{table}
\begin{figure}[h]
    \centering
    \includegraphics[width=12cm]{acc_vs_ssvsize_semimarc.png}
    \caption{不同无标签数据集规模和不平衡因子$\beta$配置下的Semi-Marc SimSiam平均准确率/宏平均召回率折线图}
    \label{acc_vs_ssvsize_semimarc}
\end{figure}

\textbf{Batch Size}

以下探讨在自监督学习中,Batch Size 对 Semi-Marc 微调网络过程的影响。从图\ref{semimarc_acc_of_batch_size}可以看出,Semi-Marc SimSiam 对 Batch Size 的变化表现出较强的稳定性,而 Semi-Marc SimCLR 的性能则呈现出逐渐下降的趋势。具体而言,Semi-Marc SimCLR 在 Batch Size 为 16 时达到了最佳性能,而 BYOL 则在 Batch Size 为 128 时表现最优。因此,本文提出的 Semi-Marc SimSiam 网络在不同 Batch Size 设置下展现了更好的稳定性。相比之下,SimCLR 和 BYOL 在训练过程中对 Batch Size 的选择较为敏感,这可能会增加网络训练人员的工作量。

\begin{figure}[h]
    \centering
    \includegraphics[width=12cm]{semimarc_acc_of_batch_size.png}
    \caption{不同Batch Size下SimSiam,SimCLR,BYOL经过Semi-Marc微调后的准确率/宏平均召回率变化图}
    \label{semimarc_acc_of_batch_size}
\end{figure}

% \begin{table}[h]
%     \centering
%     \caption{不同Batch Size下Semi-Marc SimSiam的准确率}
%     \begin{tabular}{cccccc}
%     \hline
%     \textbf{Batch Size} & \textbf{$\beta = 1$} & \textbf{$\beta = 10$} & \textbf{$\beta = 50$} & \textbf{$\beta = 100$} & \textbf{平均值} \\
%     \hline
%     16   & 0.9606 & 0.9567 & 0.9283 & 0.9031 & 0.9372 \\
%     32   & 0.9627 & 0.9604 & 0.8910 & 0.8433 & 0.9144 \\
%     64   & 0.9602 & 0.9544 & 0.9146 & 0.8946 & 0.9310 \\
%     128  & 0.9538 & 0.9585 & 0.8946 & 0.8862 & 0.9233 \\
%     256  & 0.9625 & 0.9781 & 0.8979 & 0.8652 & 0.9259 \\
%     512  & 0.9358 & 0.9448 & 0.8908 & 0.8763 & 0.9119 \\
%     \hline
%     \end{tabular}
%     \label{tab:diff_batch_size_semi_marc_acc_simsiam}
% \end{table}

% \begin{table}[h]
%     \centering
%     \caption{不同Batch Size下Semi-Marc SimCLR的准确率}
%     \begin{tabular}{cccccc}
%     \hline
%     \textbf{Batch Size} & \textbf{$\beta = 1$} & \textbf{$\beta = 10$} & \textbf{$\beta = 50$} & \textbf{$\beta = 100$} & \textbf{平均值} \\
%     \hline
%     16   & 0.9398 & 0.9458 & 0.9198 & 0.9004 & 0.9264 \\
%     32   & 0.9183 & 0.9142 & 0.8723 & 0.8335 & 0.8846 \\
%     64   & 0.8554 & 0.8585 & 0.8623 & 0.8344 & 0.8527 \\
%     128  & 0.8208 & 0.8267 & 0.8246 & 0.8117 & 0.8209 \\
%     256  & 0.7367 & 0.7419 & 0.6767 & 0.6487 & 0.7010 \\
%     512  & 0.7410 & 0.7408 & 0.6588 & 0.6165 & 0.6893 \\
%     \hline
%     \end{tabular}
%     \label{tab:diff_batch_size_semi_marc_acc_simclr}
% \end{table}

% \begin{table}[h]
%     \centering
%     \caption{不同Batch Size下Semi-Marc BYOL的准确率}
%     \begin{tabular}{cccccc}
%     \hline
%     \textbf{Batch Size} & \textbf{$\beta = 1$} & \textbf{$\beta = 10$} & \textbf{$\beta = 50$} & \textbf{$\beta = 100$} & \textbf{平均值} \\
%     \hline
%     16   & 0.9025 & 0.9092 & 0.8417 & 0.8127 & 0.8668 \\
%     32   & 0.8894 & 0.8954 & 0.8621 & 0.8319 & 0.8697 \\
%     64   & 0.9215 & 0.9129 & 0.8775 & 0.8346 & 0.8866 \\
%     128  & 0.9502 & 0.9583 & 0.9054 & 0.9050 & 0.9297 \\
%     256  & 0.9356 & 0.9306 & 0.8846 & 0.8998 & 0.9127 \\
%     500  & 0.9369 & 0.9288 & 0.8529 & 0.8381 & 0.9142 \\
%     \hline
%     \end{tabular}
%     \label{tab:diff_batch_size_semi_marc_acc_byol}
% \end{table} 
\FloatBarrier  % 阻止后续浮动体越过这条线
\section{本章小结}

本章通过结合半监督学习和MARC决策面调整的三阶段微调架构(Semi-Marc),有效地减轻了长尾分布在有标签微调阶段对SimSiam分类层的负面影响,显著提高了SimSiam在CWRU数据集和PU数据集上的性能,超越了SimCLR。此外,通过消融实验分析了编码器的“需要梯度”(Require Grad)参数、无标签数据集规模、Batch Size对模型性能的影响。
最后,作为一种通用架构,Semi-Marc在其他模型上的应用也证明了其有效性,表明该架构具有良好的普适性,能够在不同模型中提升性能。

\chapter{全文总结与展望}

\section{全文总结}
随着技术的不断发展,现代工业中的机械设备逐渐变得更加复杂,相应地,机械设备系统的故障种类也在不断增加。这导致在现实中,故障检测数据通常呈现长尾分布。然而,传统的智能故障诊断算法在处理此类长尾分布数据时,容易出现类别偏差,即模型倾向于将样本归类为头部类,而那些样本较少、但更具诊断价值和潜在危害的类别往往被忽视。此外,标注数据的获取通常需要付出高昂的时间和经济成本。

针对这一问题,本文以机械部位轴承为研究对象,首先调研了故障诊断领域及其他领域(如计算机视觉)在解决长尾学习问题时采用的方法和思路,提出了一种基于自监督学习的长尾学习故障诊断框架。接着,本文提出了基于孪生网络对比学习的自监督预训练框架,并结合半监督学习与决策面调整算法的微调框架,针对CWRU数据集和PU数据集进行了实验与分析。

本文的主要研究内容及结论如下:
\begin{enumerate}[label={(\arabic*)}]
    \item 提出了基于SimSiam简单孪生网络为骨干网络的自监督学习故障诊断框架,该框架通过对比学习进行训练。并提出了基于CMA-ES搜索最优数据增强策略的框架,为SimSiam提供了丰富有效的样本空间,解决了人工设计数据增强方法准确性和多样性不足的问题,从而使SimSiam能够学习到更深层次的语义特征。实验表明,相较于SOTA方法SimCLR和BYOL,SimSiam在CWRU数据集和PU数据集上的准确率较高仅次于SimCLR,成功减弱了长尾效应。SimSiam的优势在于无标签数据集的规模,但也存在边际效应。此外,SimSiam 在 Batch Size 变化下仍能保持稳定性能,而 SimCLR 在 Batch Size 增大时性能明显下降。
    \item 结合优化了半监督学习与MARC决策面调整相结合的微调方法(Semi-Marc),该方法与SimSiam自监督预训练组成了完整的三阶段自监督学习框架,显著减轻了长尾分布对SimSiam分类层的负面影响。实验证明,Semi-Marc方法是一种通用方法,能有效提升其他模型的性能,具有较好的普适性。经过该方法优化后SimSiam的性能成功超越了SimCLR。
\end{enumerate}

\section{后续工作展望}
虽然本文通过实验验证了该模型在轴承故障诊断中的有效性,但仍可进一步探索其适用性与稳定性。未来的研究方向如下:

\begin{enumerate}[label={(\arabic*)}]
    \item 本文提出的模型仅在轴承故障诊断中进行了验证,实际工业系统中涉及更多复杂的对象,如无人机系统等。因此,模型在其他领域中的有效性仍需进一步验证。
    \item 本文虽然在理论上提出了SimSiam数据增强模块的设计方向,但关于SimSiam简单孪生网络所需的数据增强方法尚未得出明确结论。并且,CMA-ES算法在搜索最优数据增强解时可能存在计算效率问题。未来研究可能会尝试将CMA-ES算法替换为强化学习算法,以提高效率。
    \item 本文在微调阶段未对原始数据集进行数据增强,因此,微调阶段的数据增强可能成为下一步的研究重点,例如通过生成对抗网络(GAN)进行数据增强。
\end{enumerate}

\thesisacknowledgement
在攻读硕士学位期间,衷心感谢老师、同学们的关心、支持和帮助!

\thesisappendix

% Uncomment to list all the entries of the database.
% \nocite{*}

\thesisbibliography{reference}

%
% Uncomment following codes to load bibliography database with native
% \bibliography command.
%
% \nocite{*}
% \bibliographystyle{thesis-uestc}
% \bibliography{reference}
%

\thesisaccomplish{publications}

\end{document}
